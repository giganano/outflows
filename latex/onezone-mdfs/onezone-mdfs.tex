
\documentclass[12pt]{article}
\usepackage[margin = 1in]{geometry}
\usepackage[dvipsnames]{xcolor}
\usepackage{tabularx}
\usepackage{graphicx}
\usepackage{enumitem}
\usepackage{hyperref}
% \usepackage{fancyhdr}
\usepackage[margin = 0cm]{caption}
\usepackage{wrapfig}
\usepackage{amsmath}
\usepackage{amssymb}
\usepackage{natbib}
\setlength{\bibsep}{0pt}
\hypersetup{ % all blue links
	colorlinks		= true,
	urlcolor 		= blue,
	linkcolor 		= blue,
	citecolor 		= blue
}
\setcitestyle{aysep={}, yysep={,}}
\newcommand{\aj}{AJ}
\newcommand{\aap}{A\&A}
\newcommand{\aapr}{A\&ARv}
\newcommand{\apj}{ApJ}
\newcommand{\apjl}{ApJL}
\newcommand{\apjs}{ApJS}
\newcommand{\apss}{Ap\&SS}
\newcommand{\araa}{ARA\&A}
\newcommand{\baas}{BAAS}
\newcommand{\fcp}{Fundamentals Cosmic Phys.}
\newcommand{\mnras}{MNRAS}
\newcommand{\nat}{Nature}
\newcommand{\pasa}{PASA}
\newcommand{\pasp}{PASP}
\newcommand{\ddfrac}[2]{\frac{\displaystyle{#1}}{\displaystyle{#2}}}
\newcommand{\msun}{\ensuremath{\text{M}_\odot}}
\newcommand{\scinote}[2]{\ensuremath{#1\times10^{#2}}}
\providecommand{\noopsort}[1]{}

\newcommand{\ah}{\ensuremath{\text{[$\alpha$/H]}}}
\newcommand{\timescale}[1]{\ensuremath{\tau_\text{#1}}}
\newcommand{\harmonic}[2]{\ensuremath{\bar{\tau}_\text{[#1,#2]}}}
\newcommand{\hharmonic}[3]{\ensuremath{\bar{\tau}_\text{[#1,#2,#3]}}}
\renewcommand{\tilde}[1]{\ensuremath{\widetilde{#1}}}

\begin{document}

\begin{center}
\textbf{The Shape and Centroid of Abundance Distributions for Inside-Out Star
Formation Histories in One-Zone Models}
\par\null\par
James W. Johnson
\par\null\par
\rule[0.7\baselineskip]{0.5\textwidth}{0.4pt}
\end{center}

\par\noindent
The goal here is to obtain an analytic or parametric expression for the
metallicity distribution function (MDF) given the ``inside-out'' star formation
history (SFH) from~\citet{Johnson2021}:
\begin{equation}
\dot{M}_\star(t) \propto (1 - e^{-t / \timescale{rise}})
e^{-t / \timescale{sfh}}.
\label{eq:insideout_sfh}
\end{equation}
The physically interesting advantage of this parameterization over the classic
``linear-times-exponential''~$t e^{-t / \timescale{sfh}}$ form is that the
rise timescale~\timescale{rise}~gives one some control over how long the SFH is
increasing independent of~\timescale{sfh}.
In the interest of analytic/parametric solutions, I assume that the star
formation efficiency (SFE) timescale~$\tau_\star \equiv M_\text{g} /
\dot{M}_\star$ and the outflow mass loading factor~$\eta \equiv
\dot{M}_\text{out} / \dot{M}_\star$ are both constant in time.
\par
In the interest of simplicity, I have so far focused on alpha elements with
negligible yields from Type Ia supernovae (e.g., O and Mg).
Provided a solution for the abundance evolution~$Z_\alpha(t)$, the MDF can be
expressed according to
\begin{equation}
\frac{dN}{dZ_\alpha} = \frac{dN}{dt} \bigg/ \frac{dZ_\alpha}{dt}
\propto \frac{\dot{M}_\star}{\dot{Z}_\alpha}.
\label{eq:dndz}
\end{equation}
Therefore, I will need not only~$Z_\alpha(t)$ but its time-derivative as well.
The MDF in~\ah~can conversely be expressed as:
\begin{equation}
\frac{dN}{d\ah} = \frac{dN}{dZ_\alpha} \bigg/ \frac{d\ah}{dZ_\alpha},
\end{equation}
where substituting the conventional notation~$\ah \equiv \log_{10}(Z_\alpha /
Z_{\alpha,\odot})$ then yields
\begin{equation}
\frac{dN}{d\ah} = \ln 10 \left(\frac{Z_\alpha}{Z_{\alpha,\odot}}\right)
\frac{dN}{dZ_\alpha}
\propto \frac{\dot{M}_\star}{\dot{Z}_\alpha / Z_\alpha},
\label{eq:dndlogz}
\end{equation}
where I have neglected the normalizing prefactors~$\ln 10$ and the solar
alpha abundance~$Z_{\alpha,\odot}$ as the final term is a proportionality which
will be re-normalized anyway.
\par
Given an analytic or parametric expression for the MDF, the position of its
maximum can be determined by taking the second derivative~$d^2 N / d Z_\alpha$
or~$d^2 N / d \ah^2$ and setting it equal to zero.
This procedure will produce a family of solutions for regions of parameter
space in which there is a meaningful maximum in the predicted MDF.
In other cases, the MDF will be monotonically rising with both~$Z_\alpha$ and
time, and the position of the maximum can be reasonably approximated as the
value of~$Z_\alpha$ at the present-day predicted by the model.
With these expressions, I can identify which parameters have the most power
over determining the centroid of the MDF, which I can then anchor to the
observed metallicity gradient in the Galaxy as the observational benchmark to
construct parameter choices for models with no mass loading (i.e.,~$\eta = 0$).

\par\null\par\noindent
\textbf{Analytic solution to~$Z_\alpha(t)$}
\par\noindent
The expression for~$\dot{M}_\star$ is given by the assumption of equation
\ref{eq:insideout_sfh}, so now I must solve for~$Z_\alpha(t)$ using arguments
similar to those of~\citet*{Weinberg2017}.
For alpha elements,~\citet*{Weinberg2017} express the rate of change of the
mass present in the ISM as
\begin{equation}
\dot{M}_\alpha = y_\alpha \dot{M}_\star - Z_\alpha \dot{M}_\star (1 + \eta - r),
\label{eq:mdot_alpha}
\end{equation}
where~$r$ is a term which accounts for the return of stellar envelopes back to
the ISM.
\par
Based on the definition of~$Z_\alpha$, the quotient rule implies that its
time-derivative can be expressed as
\begin{subequations}\begin{align}
\dot{Z}_\alpha &= \frac{
	M_\text{g} \dot{M}_\alpha - M_\alpha \dot{M}_\text{g}
}{
	M_\text{g}^2
}
\\
&= \frac{
	M_\text{g} (y_\alpha \dot{M}_\star - Z_\alpha \dot{M}_\star (1 + \eta - r))
	- M_\alpha \ddot{M}_\star \tau_\star
}{
	M_\text{g}^2
}
\\
&= y_\alpha \frac{\dot{M}_\star}{M_\text{g}} -
Z_\alpha \frac{\dot{M}_\star}{M_\text{g}}(1 + \eta - r) -
Z_\alpha \frac{\ddot{M}_\star}{M_\text{g}} \tau_\star
\\
&= \frac{y_\alpha}{\tau_\star} -
Z_\alpha \left(\frac{1 + \eta - r}{\tau_\star} +
\frac{\ddot{M}_\star}{\dot{M}_\star}\right).
\label{eq:dzdt}
\end{align}\end{subequations}
The next step is to differentiate the SFH with time to compute
$\ddot{M}_\star / \dot{M}_\star$. Taking~$A$ to denote the overall normalizing
factor:
\begin{subequations}\begin{align}
\ddot{M}_\star &= \frac{d}{dt}
A(1 - e^{-t / \timescale{rise}}) e^{-t / \timescale{sfh}}
\\
&= A \frac{1}{\timescale{rise}}e^{-t / \timescale{rise}}
e^{-t / \timescale{sfh}} +
A(1 - e^{-t / \timescale{rise}}) \frac{-1}{\timescale{sfh}}
e^{-t / \timescale{sfh}}
\\
&= A(1 - e^{-t / \timescale{rise}})e^{-t / \timescale{sfh}}
\left(\frac{
	e^{-t / \timescale{rise}}
}{
	\timescale{rise}(1 - e^{-t / \timescale{rise}})
} - \frac{1}{\timescale{sfh}}
\right)
\\
\implies \frac{\ddot{M}_\star}{\dot{M}_\star} &= \frac{
	e^{-t / \timescale{rise}}
}{
	\timescale{rise}(1 - e^{-t / \timescale{rise}})
} - \frac{1}{\timescale{sfh}}.
\label{eq:mddotstar-over-mdotstar}
\end{align}\end{subequations}
At this point, it is straightforward to compute the equilibrium alpha element
abundance for this SFH by plugging equation~\ref{eq:mddotstar-over-mdotstar}
into equation~\ref{eq:dzdt} and setting~$\dot{Z}_\alpha = 0$.
This procedure results in the following expression:
\begin{equation}
Z_{\alpha,\text{eq}} = \ddfrac{
	y_\alpha
}{
	1 + \eta - r + \frac{
		\tau_\star e^{-t / \timescale{rise}}
	}{
		\timescale{rise}(1 - e^{-t / \timescale{rise}})
	} - \frac{\tau_\star}{\timescale{sfh}}
}.
\label{eq:zalpha-eq}
\end{equation}
Equation~\ref{eq:zalpha-eq} indicates that unlike the simpler SFHs like a
constant or single exponential, the equilibrium abundance is not uniform in
time.
Instead, the term depending on~\timescale{rise} is infinite at~$t = 0$,
increases at early times, and then approaches the solution for a single
exponential once~$t \gg \timescale{rise}$.
As a sanity check, equation~\ref{eq:zalpha-eq} indeed reduces to the
equilibrium abundance for a single exponential SFH found
by~\citet*{Weinberg2017} if I let~$\timescale{rise} \rightarrow 0$.
\par
Plugging this into equation~\ref{eq:dzdt} above yields the following linear
ODE for~$Z_\alpha$:
\begin{equation}
\dot{Z}_\alpha + Z_\alpha \left(\frac{1}{\timescale{dep}} +
\frac{
	e^{-t / \timescale{rise}}
}{
	\timescale{rise}(1 - e^{-t / \timescale{rise}})
} - \frac{1}{\timescale{sfh}}\right) = \frac{y_\alpha}{\tau_\star},
\end{equation}
where I have substituted for the depletion time, defined as
$\timescale{dep} \equiv \tau_\star / (1 + \eta - r)$.
$\timescale{dep}$ quantifies the e-folding timescale on which the ISM would
decline due to both star formation and mass loading, if present.
For notational convenience, I define the function~$f(t)$ to denote the
multiplicative factor on~$Z_\alpha$:
\begin{equation}
f(t) \equiv \frac{1}{\timescale{dep}} + \frac{
	e^{-t / \timescale{rise}}
}{
	\timescale{rise}(1 - e^{-t / \timescale{rise}})
} - \frac{1}{\timescale{sfh}},
\end{equation}
such that the general solution for~$Z_\alpha$ can be expressed as
\begin{equation}
Z_\alpha(t) = \exp\left(-\int f(t') dt'\right)\left(
\int_0^t \exp\left(\int f(t') dt'\right) \frac{y_\alpha}{\tau_\star} dt' + C
\right),
\label{eq:za-linear-ode}
\end{equation}
where the constant~$C$ will later be assigned such that the initial
condition~$Z_\alpha(t = 0) = 0$ is satisfied.
At this point, it is helpful to zoom in on the integral of~$f(t)$.
Its term depending on~$\timescale{rise}$ can be integrated with a series of
variable substitutions:
\begin{subequations}\begin{align}
\int \frac{
	e^{-t / \timescale{rise}}
}{
	\timescale{rise}(1 - e^{-t / \timescale{rise}})
} dt
&= \int \frac{e^{-u}}{1 - e^{-u}} du
\qquad \left(u = \frac{t}{\timescale{rise}};~~du = \frac{1}{\timescale{rise}}
dt\right)
\\
&= \int \frac{dv}{v} \qquad \left(v = 1 - e^{-u};~~dv = e^{-u} du\right)
\\
&= \ln v
\\
&= \ln (1 - e^{-u})
\\
&= \ln (1 - e^{-t / \timescale{rise}})
\\
\implies \int f(t) dt
&= \frac{t}{\timescale{dep}} + \ln (1 - e^{-t / \timescale{rise}}) -
\frac{t}{\timescale{sfh}}.
\end{align}\end{subequations}
Solving equation~\ref{eq:za-linear-ode} amounts to integrating the sum of a
handful of exponentials with prefactors that depend on~\timescale{rise},
\timescale{dep}, and~\timescale{sfh} in a non-trivial way.
The full derivation is attached, and the final solution is given by
\begin{equation}
\begin{split}
Z_\alpha(t) &= \frac{1}{1 - e^{-t / \timescale{rise}}}
\left(\frac{y_\alpha}{1 + \eta - r}\right)
\bigg[\frac{
	\timescale{sfh}
}{
	\timescale{sfh} - \timescale{dep}
} \left(
1 - \exp\left(-t\frac{
	\timescale{sfh} - \timescale{dep}
}{
	\timescale{sfh}\timescale{dep}
}\right)
\right) -
\\
&\qquad \frac{
	\timescale{sfh}\timescale{rise}
}{
	\timescale{sfh}\timescale{rise} - \timescale{dep}\timescale{rise} -
	\timescale{sfh}\timescale{dep}
} \left(e^{-t / \timescale{rise}} -
\exp\left(-t
\frac{
	\timescale{sfh} - \timescale{dep}
}{
	\timescale{sfh}\timescale{dep}
}
\right)
\right)\bigg].
\end{split}
\end{equation}
Adopting the harmonic timescale notation~$\harmonic{X}{Y} = \timescale{Y}
\timescale{X} / (\timescale{Y} - \timescale{X})$ of~\citet*{Weinberg2017}
simplifies the expression:
\begin{equation}
\begin{split}
Z_\alpha(t) &= \frac{1}{1 - e^{-t / \timescale{rise}}} \left( \frac{
	y_\alpha
}{
	1 + \eta - r
}\right) \bigg[
\frac{\harmonic{dep}{sfh}}{\timescale{dep}}
\left(1 - e^{-t / \harmonic{dep}{sfh}}\right) -
\\
&\qquad \frac{\hharmonic{dep}{rise}{sfh}}{\timescale{dep}}
\left(e^{-t / \timescale{rise}} - e^{-t / \harmonic{dep}{sfh}}\right)
\bigg]
\end{split}
\label{eq:zalpha}
\end{equation}
\par
At~$t = 0$, the factor~$1 / (1 - e^{-t / \timescale{rise}})$ is infinite,
but the factor enclosed in square brackets is zero.
Therefore, a physical solution exists if and only if the integration
constant~$C$, included in the full derivation of this expression (see equation
\ref{eq:zalpha-full}), is equal to zero.
Because this results in the~$\infty \times 0$ indeterminate form, I can simply
define the abundance to be zero such that the boundary condition
of~$Z_\alpha(t = 0) = 0$ is satisfied.

\begin{figure*}
\centering
\includegraphics[scale = 0.52]{vartimescales.pdf}
\includegraphics[scale = 0.52]{varyieldeta.pdf}
\caption{
Analytically computed evolution in the oxygen abundance according to equation
\ref{eq:zalpha}.
\textbf{Left}: For~$y_\alpha = 0.015$ and~$\eta = 2.5$, each curve denotes a
different choice of some timescale.
With black visualizing a fiducial choice of parameters of (\timescale{rise},
\timescale{sfh},~$\tau_\star$) = (2 Gyr, 6 Gyr, 5 Gyr), crimson shows a short
rise timescale (i.e.,~$\timescale{rise} = 1$ Gyr), lime green shows a more
extended SFH (i.e.,~$\timescale{sfh} = 10$ Gyr), and blue shows a higher SFE
(i.e.,~$\tau_\star = 2$ Gyr).
\textbf{Right}: For the fiducial choice of timescales in the right-hand panel,
each curve denotes a different choice of~$y_\alpha$ and~$\eta$ as denoted in
the legend.
For each choice of~$y_\alpha$, the value of~$\eta$ is chosen such that the
ratio~$y_\alpha / (1 + \eta - r)$ is approximately constant, and vice versa in
the case of the blue line where I choose~$\eta = 0$ and compute the
corresponding value of~$y_\alpha$.
}
\label{fig:analytic-evolution}
\end{figure*}

The left panel of Fig.~\ref{fig:analytic-evolution} visualizes the evolution of
the alpha element abundances according to equation~\ref{eq:zalpha} under
different choices of timescales.
% A shorter rise timescale has the effect of raising the abundances overall
\par
The right panel of Fig.~\ref{fig:analytic-evolution} visualizes the enrichment
history for the fiducial choice of timescales in the left panel, but with
different values of~$y_\alpha$ and~$\eta$, selected such that the value of
$y_\alpha / (1 + \eta - r)$ is approximately constant.
\par\null\par\noindent
\textbf{The Time-Derivative of~$Z_\alpha(t)$}
\par\noindent
The differentiate~$Z_\alpha(t)$ with time, it compactifies notation to define
some function~$g(t)$ denoting the term in square brackets in equation
\ref{eq:zalpha}:
\begin{equation}
g(t) \equiv \frac{\harmonic{dep}{sfh}}{\timescale{dep}}
\left(1 - e^{-t / \harmonic{dep}{sfh}}\right) -
\frac{\hharmonic{dep}{rise}{sfh}}{\timescale{dep}}
\left(e^{-t / \timescale{rise}} - e^{-t / \harmonic{dep}{sfh}}\right),
\label{eq:g}
\end{equation}
such that the expression for~$Z_\alpha$ reduces to
\begin{equation}
Z_\alpha(t) = \frac{1}{1 - e^{-t / \timescale{rise}}}
\left(\frac{y_\alpha}{1 + \eta - r}\right) g(t),
\end{equation}
and its time-derivative can then be obtained with product-rule:
\begin{subequations}\begin{align}
\dot{Z}_\alpha(t) &= \left(\frac{y_\alpha}{1 + \eta - r}\right)\left[
\frac{
	-e^{-t / \timescale{rise}}
}{
	\timescale{rise} \left(1 - e^{-t / \timescale{rise}}\right)^2
} g(t) + \frac{1}{1 - e^{-t / \timescale{rise}}} \dot{g}(t)
\right]
\label{eq:zdotalpha}
\\
&= \frac{1}{1 - e^{-t / \timescale{rise}}}
\left(\frac{y_\alpha}{1 + \eta - r}\right) g(t)
\left[
\frac{
	-e^{-t / \timescale{rise}}
}{
	\timescale{rise} \left(1 - e^{-t / \timescale{rise}}\right)
} + \frac{\dot{g}(t)}{g(t)}\right]
\\
\implies \frac{\dot{Z}_\alpha(t)}{Z_\alpha(t)} &=
\frac{\dot{g}(t)}{g(t)} - \frac{
	-e^{-t / \timescale{rise}}
}{
	\timescale{rise} \left(1 - e^{-t / \timescale{rise}}\right)
}.
\label{eq:zdotalpha_over_zalpha}
\end{align}\end{subequations}
The time-derivative of~$g(t)$ is straight-forward:
\begin{subequations}\begin{align}
\begin{split}
\dot{g}(t) &= \frac{\harmonic{dep}{sfh}}{\timescale{dep}}
\left(0 - e^{-t / \harmonic{dep}{sfh}} \frac{-1}{\harmonic{dep}{sfh}}\right) -
\\
&\qquad \frac{\hharmonic{dep}{rise}{sfh}}{\timescale{dep}}
\left(e^{-t / \timescale{rise}}\frac{-1}{\timescale{rise}} -
e^{-t / \harmonic{dep}{sfh}} \frac{-1}{\harmonic{dep}{sfh}}\right)
\end{split}
\\
&= \frac{e^{-t / \harmonic{dep}{sfh}}}{\timescale{dep}} +
\frac{\hharmonic{dep}{rise}{sfh}}{\timescale{dep}}
\left(\frac{e^{-t / \timescale{rise}}}{\timescale{rise}} -
\frac{e^{-t / \harmonic{dep}{sfh}}}{\harmonic{dep}{sfh}}\right)
\label{eq:gdot}
\\
\implies \frac{\dot{g}(t)}{g(t)} &= \ddfrac{
	e^{-t / \harmonic{dep}{sfh}} + \hharmonic{dep}{rise}{sfh}
	\left(\frac{e^{-t / \timescale{rise}}}{\timescale{rise}} -
	\frac{e^{-t / \harmonic{dep}{sfh}}}{\harmonic{dep}{sfh}}\right)
}{
	\harmonic{dep}{sfh}\left(1 - e^{-t / \harmonic{dep}{sfh}}\right) -
	\hharmonic{dep}{rise}{sfh}\left(e^{-t / \timescale{rise}} -
	e^{-t / \harmonic{dep}{sfh}}\right)
}.
\label{eq:gdot_over_g}
\end{align}\end{subequations}
Between equations~\ref{eq:zdotalpha},~\ref{eq:zdotalpha_over_zalpha},
\ref{eq:gdot}, and~\ref{eq:gdot_over_g}, the full solution
to~$\dot{Z}_\alpha(t)$ and~$\dot{Z}_\alpha(t) / Z_\alpha(t)$ is specified.

\par\null\par\noindent
\textbf{The Second Time-Derivative of~$Z_\alpha(t)$}
\par\noindent
Differentiating~$\dot{Z}_\alpha(t)$ with time follows from
equation~\ref{eq:zdotalpha}:
\begin{subequations}\begin{align}
\begin{split}
\ddot{Z}_\alpha(t) &= \left(\frac{y_\alpha}{1 + \eta - r}\right)\Bigg[
\left(\frac{
	e^{-t / \timescale{rise}}
}{
	\timescale{rise}^2 \left(1 - e^{-t / \timescale{rise}}\right)^2
} + \frac{
	2e^{-2t / \timescale{rise}}
}{
	\timescale{rise}^2 \left(1 - e^{-t / \timescale{rise}}\right)^3
}\right)g(t) -
\\
&\qquad \frac{
	e^{-t / \timescale{rise}}
}{
	\timescale{rise} \left(1 - e^{-t / \timescale{rise}}\right)
} \dot{g}(t) + \frac{
	-e^{-t / \timescale{rise}}
}{
	\timescale{rise} \left(1 - e^{-t / \timescale{rise}}\right)^2
} \dot{g}(t) + \frac{1}{1 - e^{-t / \timescale{rise}}} \ddot{g}(t) \Bigg]
\end{split}
\\
\begin{split}
&= \left(\frac{y_\alpha}{1 + \eta - r}\right) \Bigg[
\left(\frac{
	e^{-t / \timescale{rise}}
}{
	\left(1 - e^{-t / \timescale{rise}}\right)^2
} + \frac{
	2e^{-2t / \timescale{rise}}
}{
	\left(1 - e^{-t / \timescale{rise}}\right)^3
}\right) \frac{g(t)}{\timescale{rise}^2} -
\\
&\qquad \left(\frac{
	2e^{-t / \timescale{rise}} + e^{-2t / \timescale{rise}}
}{
	\timescale{rise} \left(1 - e^{-t / \timescale{rise}}\right)^2
}\right) \dot{g}(t) + \frac{1}{1 - e^{-t / \timescale{rise}}} \ddot{g}(t)
\Bigg],
\end{split}
\end{align}\end{subequations}
and differentiating~$\dot{g}(t)$ with time follows from
equation~\ref{eq:gdot}:
\begin{equation}
\ddot{g}(t) = \frac{
	-e^{-t / \harmonic{dep}{sfh}}
}{
	\timescale{dep}\harmonic{dep}{sfh}
} - \frac{\hharmonic{dep}{rise}{sfh}}{\timescale{dep}}\left(\frac{
	e^{-t / \timescale{rise}}
}{
	\timescale{rise}^2
} - \frac{
	e^{-t / \harmonic{dep}{sfh}}
}{
	\harmonic{dep}{sfh}^2
}\right)
\end{equation}

\newpage
\bibliographystyle{mnras}
\bibliography{onezone-mdfs}

\newpage
\noindent
\textbf{Full Solution to Equation~\ref{eq:za-linear-ode}}

\begin{subequations}\begin{align}
\begin{split} % a
Z_\alpha(t) &= \exp\left(
\frac{-t}{\timescale{dep}} - \ln (1 - e^{-t / \timescale{rise}}) +
\frac{t}{\timescale{sfh}}
\right)
\\
&\qquad \left[
\int_0^t \exp\left(
\frac{t'}{\timescale{dep}} + \ln (1 - e^{-t / \timescale{rise}}) -
\frac{t'}{\timescale{sfh}}
\right)
\frac{y_\alpha}{\tau_\star} dt' + C
\right]
\end{split}
\\
\begin{split} % b
&= \frac{1}{1 - e^{-t / \timescale{rise}}}
\exp \left(-t\frac{
	\timescale{sfh} - \timescale{dep}
}{
	\timescale{sfh}\timescale{dep}
}\right) \frac{y_\alpha}{\tau_\star}
\\
&\qquad \left[ \int_0^t (1 - e^{-t' / \timescale{rise}}) \exp \left(
t' \frac{\timescale{sfh} - \timescale{dep}}{\timescale{sfh} \timescale{dep}}
\right)dt' + C\right]
\end{split}
\\
\begin{split} % c
&= \frac{1}{1 - e^{-t / \timescale{rise}}}
\exp \left(-t\frac{
	\timescale{sfh} - \timescale{dep}
}{
	\timescale{sfh}\timescale{dep}
}\right) \frac{y_\alpha}{\tau_\star}
\\
&\qquad \left[ \int_0^t \left(
\exp\left(t' \frac{
	\timescale{sfh} - \timescale{dep}
}{
	\timescale{sfh}\timescale{dep}
}\right)
- \exp\left(
t' \frac{
	\timescale{sfh} - \timescale{dep}
}{
	\timescale{sfh}\timescale{dep}
} - \frac{t}{\timescale{rise}}
\right)
\right) dt' + C\right]
\end{split}
\\
\begin{split} % d
&= \frac{1}{1 - e^{-t / \timescale{rise}}}
\exp \left(-t\frac{
	\timescale{sfh} - \timescale{dep}
}{
	\timescale{sfh}\timescale{dep}
}\right) \frac{y_\alpha}{\tau_\star} \bigg[ \int_0^t
\exp\left(t'\frac{
	\timescale{sfh} - \timescale{dep}
}{
	\timescale{sfh}\timescale{dep}
} \right) dt' -
\\
&\qquad
\int_0^t \exp\left(t'\frac{
	\timescale{sfh}\timescale{rise} - \timescale{dep}\timescale{rise} -
	\timescale{sfh}\timescale{dep}
}{
	\timescale{sfh}\timescale{dep}\timescale{rise}
}\right) dt' + C \bigg]
\end{split}
\\
\begin{split} % e
&= \frac{1}{1 - e^{-t / \timescale{rise}}}
\exp \left(-t\frac{
	\timescale{sfh} - \timescale{dep}
}{
	\timescale{sfh}\timescale{dep}
}\right) \frac{y_\alpha}{\tau_\star} \bigg[\frac{
	\timescale{sfh}\timescale{dep}
}{
	\timescale{sfh} - \timescale{dep}
} \exp \left(t' \frac{
	\timescale{sfh} - \timescale{dep}
}{
	\timescale{sfh}\timescale{dep}
}\right) \bigg|_0^t -
\\
&\qquad \frac{
	\timescale{sfh}\timescale{dep}\timescale{rise}
}{
	\timescale{sfh}\timescale{rise} - \timescale{dep}\timescale{rise} -
	\timescale{sfh}\timescale{dep}
} \exp\left(
t'\frac{
	\timescale{sfh}\timescale{rise} - \timescale{dep}\timescale{rise} -
	\timescale{sfh}\timescale{dep}
}{
	\timescale{sfh}\timescale{dep}\timescale{rise}
}
\right)\bigg|_0^t + C \bigg]
\end{split}
\\
\begin{split} % f
&= \frac{1}{1 - e^{-t / \timescale{rise}}}
\exp \left(-t\frac{
	\timescale{sfh} - \timescale{dep}
}{
	\timescale{sfh}\timescale{dep}
}\right) \frac{y_\alpha}{\tau_\star} \bigg[\frac{
	\timescale{sfh}\timescale{dep}
}{
	\timescale{sfh} - \timescale{dep}
} \left(
\exp\left(
t\frac{
	\timescale{sfh} - \timescale{dep}
}{
	\timescale{sfh}\timescale{dep}
}\right) - 1\right) -
\\
&\qquad \frac{
	\timescale{sfh}\timescale{dep}\timescale{rise}
}{
	\timescale{sfh}\timescale{rise} - \timescale{dep}\timescale{rise} -
	\timescale{sfh}\timescale{dep}
} \left(
\exp\left(t\frac{
	\timescale{sfh}\timescale{rise} - \timescale{dep}\timescale{rise} -
	\timescale{sfh}\timescale{dep}
}{
	\timescale{sfh}\timescale{dep}\timescale{rise}
}\right) - 1\right)
\\
&\qquad + C\bigg]
\end{split}
\\
\begin{split} % g
&= \frac{1}{1 - e^{-t / \timescale{rise}}}
\left(\frac{y_\alpha}{\tau_\star}\right)
\bigg[\frac{
	\timescale{sfh}\timescale{dep}
}{
	\timescale{sfh} - \timescale{dep}
} \left(
1 - \exp\left(-t\frac{
	\timescale{sfh} - \timescale{dep}
}{
	\timescale{sfh}\timescale{dep}
}\right)
\right) -
\\
&\qquad \frac{
	\timescale{sfh}\timescale{dep}\timescale{rise}
}{
	\timescale{sfh}\timescale{rise} - \timescale{dep}\timescale{rise} -
	\timescale{sfh}\timescale{dep}
} \left(e^{-t / \timescale{rise}} -
\exp\left(-t
\frac{
	\timescale{sfh} - \timescale{dep}
}{
	\timescale{sfh}\timescale{dep}
}
\right)
\right) + C\bigg]
\end{split}
\\
\begin{split} % h
&= \frac{1}{1 - e^{-t / \timescale{rise}}}
\left(\frac{y_\alpha}{1 + \eta - r}\right)
\bigg[\frac{
	\timescale{sfh}
}{
	\timescale{sfh} - \timescale{dep}
} \left(
1 - \exp\left(-t\frac{
	\timescale{sfh} - \timescale{dep}
}{
	\timescale{sfh}\timescale{dep}
}\right)
\right) -
\\
&\qquad \frac{
	\timescale{sfh}\timescale{rise}
}{
	\timescale{sfh}\timescale{rise} - \timescale{dep}\timescale{rise} -
	\timescale{sfh}\timescale{dep}
} \left(e^{-t / \timescale{rise}} -
\exp\left(-t
\frac{
	\timescale{sfh} - \timescale{dep}
}{
	\timescale{sfh}\timescale{dep}
}
\right)
\right) + C\bigg].
\end{split}
\label{eq:zalpha-full}
\end{align}\end{subequations}

\newpage
\noindent
\textbf{Useful Identities}
\begin{subequations}\begin{align}
\frac{d}{dt}\left(1 - e^{-t / \timescale{rise}}\right)^{-1} &=
-\left(1 - e^{-t / \timescale{rise}}\right)^{-2}
\frac{d}{dt}\left(1 - e^{-t / \timescale{rise}}\right)
\\
&= \frac{-1}{\left(1 - e^{-t / \timescale{rise}}\right)^2}
\left(0 - e^{-t / \timescale{rise}}\left(\frac{-t}{\timescale{rise}}\right)
\right)
\\
&= \frac{
	-e^{-t / \timescale{rise}}
}{
	\timescale{rise} \left(1 - e^{-t / \timescale{rise}}\right)^2
}
\end{align}\end{subequations}
\begin{subequations}\begin{align}
\begin{split}
\frac{d}{dt}\left(\frac{
	-e^{-t / \timescale{rise}}
}{
	\left(1 - e^{-t / \timescale{rise}}\right)^2
}\right) &= \frac{1}{(1 - e^{-t / \timescale{rise}})^4}\bigg[
\left(1 - e^{-t / \timescale{rise}}\right)^2
\left(-e^{-t / \timescale{rise}}\right)
\left(-1 / \timescale{rise}\right) +
\\
&\qquad e^{-t / \timescale{rise}}
2\left(1 - e^{-t / \timescale{rise}}\right)
\left(0 - e^{-t / \timescale{rise}}\left(-1 / \timescale{rise}\right)\right)
\bigg]
\end{split}
\\
&= \frac{
	e^{-t / \timescale{rise}}
	\left(1 - e^{-t / \timescale{rise}}\right)^2 +
	2e^{-2t / \timescale{rise}}
	\left(1 - e^{-t / \timescale{rise}}\right)
}{
	\timescale{rise} \left(1 - e^{-t / \timescale{rise}}\right)^4
}
\\
&= \frac{
	e^{-t / \timescale{rise}}
}{
	\timescale{rise} \left(1 - e^{-t / \timescale{rise}}\right)^2
} + \frac{
	2 e^{-2t / \timescale{rise}}
}{
	\timescale{rise} \left(1 - e^{-t / \timescale{rise}}\right)^3
}
\end{align}\end{subequations}

\end{document}




% \textbf{Time-Derivative of~$Z_\alpha(t)$}
% \par\noindent
% Taking the time-derivative of equation~\ref{eq:zalpha} first and foremost
% requires product rule.
% It is helpful to split this process up into pieces:
% \begin{subequations}\begin{align}
% \frac{d}{dt} (1 - e^{-t / \timescale{rise}})^{-1} &=
% -(1 - e^{-t / \timescale{rise}})^{-2}\frac{d}{dt}e^{-t / \timescale{rise}}
% \\
% &= \frac{
% 	e^{-t / \timescale{rise}}
% }{
% 	\timescale{rise} (1 - e^{-t / \timescale{rise}})^2
% }.
% \end{align}\end{subequations}
% To save space, I define the function~$g(t)$ to denote the term in square
% brackets in equation~\ref{eq:zalpha}.
% Its time derivative:
% \begin{subequations}\begin{align}
% \begin{split} % a
% \dot{g}(t) &= \frac{\timescale{sfh}}{\timescale{sfh} - \timescale{dep}}
% \exp \left( -t \frac{
% 	\timescale{sfh} - \timescale{dep}
% }{
% 	\timescale{sfh}\timescale{dep}
% }\right)\frac{
% 	\timescale{sfh} - \timescale{dep}
% }{
% 	\timescale{sfh}\timescale{dep}
% } - 
% \\
% &\qquad \frac{
% 	\timescale{sfh}\timescale{rise}
% }{
% 	\timescale{sfh}\timescale{rise} - \timescale{dep}\timescale{rise} -
% 	\timescale{sfh}\timescale{dep}
% } \bigg(
% \frac{-1}{\timescale{rise}} e^{-t / \timescale{rise}} +
% \\
% &\qquad
% \exp \left( -t \frac{
% 	\timescale{sfh} - \timescale{dep}
% }{
% 	\timescale{sfh}\timescale{dep}
% }
% \right)
% \frac{
% 	\timescale{sfh} - \timescale{dep}
% }{
% 	\timescale{sfh}\timescale{dep}
% }
% \bigg)
% \end{split}
% \\
% \begin{split} % b
% &= \frac{1}{\timescale{dep}}\exp\left( -t \frac{
% 	\timescale{sfh} - \timescale{dep}
% }{
% 	\timescale{sfh}\timescale{dep}
% }\right) + \frac{
% 	\timescale{sfh}
% }{
% 	\timescale{sfh}\timescale{rise} - \timescale{dep}\timescale{rise} -
% 	\timescale{sfh}\timescale{dep}
% } e^{-t / \timescale{rise}} -
% \\
% &\qquad \frac{
% 	\timescale{rise}\timescale{sfh} - \timescale{rise}\timescale{dep}
% }{
% 	\timescale{sfh}\timescale{rise}\timescale{dep} - \timescale{dep}^2
% 	\timescale{rise} - \timescale{sfh}\timescale{dep}^2
% } \exp \left( -t \frac{
% 	\timescale{sfh} - \timescale{dep}
% }{
% 	\timescale{sfh}\timescale{dep}
% }
% \right)
% \end{split}
% \\
% \begin{split} % c
% &= \exp\left(-t \frac{
% 	\timescale{sfh} - \timescale{dep}
% }{
% 	\timescale{sfh}\timescale{dep}
% }\right) \bigg[
% \frac{1}{\timescale{dep}} - \frac{
% 	\timescale{rise}\timescale{sfh} - \timescale{rise}\timescale{dep}
% }{
% 	\timescale{sfh}\timescale{rise}\timescale{dep} - \timescale{dep}^2
% 	\timescale{rise} - \timescale{sfh}\timescale{dep}^2
% } +
% \\
% &\qquad \frac{
% 	\timescale{sfh}
% }{
% 	\timescale{sfh}\timescale{rise} - \timescale{dep}\timescale{rise} -
% 	\timescale{sfh}\timescale{dep}
% } \exp \left(t \frac{
% 	\timescale{rise}\timescale{sfh} - \timescale{dep}\timescale{rise} -
% 	\timescale{sfh}\timescale{dep}
% }{
% 	\timescale{rise}\timescale{sfh}\timescale{dep}
% }\right) \bigg]
% \end{split}
% \\
% \begin{split} % d
% &= \exp\left(-t \frac{
% 	\timescale{sfh} - \timescale{dep}
% }{
% 	\timescale{sfh}\timescale{dep}
% }\right) \frac{
% 	\timescale{sfh}
% }{
% 	\timescale{sfh}\timescale{rise} - \timescale{dep}\timescale{rise} -
% 	\timescale{sfh}\timescale{dep}
% }
% \\
% &\qquad \left[\exp\left(t \frac{
% 	\timescale{rise}\timescale{sfh} - \timescale{dep}\timescale{rise} -
% 	\timescale{sfh}\timescale{dep}
% }{
% 	\timescale{rise}\timescale{sfh}\timescale{dep}
% }\right)
% - 1\right].
% \end{split}
% \end{align}\end{subequations}
% At this point, it simplifies notation to define
% \begin{equation}
% \hharmonic{dep}{sfh}{rise} \equiv \frac{
% 	\timescale{rise}\timescale{sfh}\timescale{dep}
% }{
% 	\timescale{sfh}\timescale{rise} - \timescale{dep}\timescale{rise} -
% 	\timescale{sfh}\timescale{dep}
% } = \left( \frac{1}{\timescale{dep}} - \frac{1}{\timescale{sfh}} -
% \frac{1}{\timescale{rise}}\right)^{-1},
% \end{equation}
% which is reminiscent of the harmonic timescales~$\harmonic{X}{Y}$ seen
% in~\citet*{Weinberg2017}, but for three timescales instead of two.
% Adopting this notation yields the following expression for~$\dot{g}(t)$:
% \begin{equation}
% \begin{split}
% \dot{g}(t) &= \frac{
% 	\hharmonic{dep}{rise}{sfh}
% }{
% 	\timescale{rise}\timescale{dep}
% } e^{-t / \harmonic{dep}{sfh}} \left(e^{t /
% \hharmonic{dep}{rise}{sfh}} - 1\right)
% \\
% &= \frac{
% 	\hharmonic{dep}{rise}{sfh}
% }{
% 	\timescale{rise}\timescale{dep}
% } \left(e^{-t / \timescale{rise}} - e^{-t / \harmonic{dep}{sfh}}\right).
% \end{split}
% \end{equation}
% I can now write the full expression for~$\dot{Z}_\alpha(t)$:
% \begin{subequations}\begin{align}
% \dot{Z}_\alpha(t) &= \left(\frac{y_\alpha}{1 + \eta - r}\right)
% \left[
% \frac{e^{-t / \timescale{rise}}}{
% 	\timescale{rise} (1 - e^{-t / \timescale{rise}})^2
% } g(t) + \frac{1}{1 - e^{-t / \timescale{rise}}} \dot{g}(t)
% \right]
% \\
% &= \left(\frac{y_\alpha}{1 + \eta - r}\right)
% \frac{1}{1 - e^{-t / \timescale{rise}}} \left[
% \dot{g}(t) + \frac{
% 	e^{-t / \timescale{rise}}
% }{
% 	\timescale{rise} (1 - e^{-t / \timescale{rise}})
% } g(t)
% \right],
% \end{align}\end{subequations}
% and an expression for~$\dot{Z}_\alpha(t) / Z_\alpha(t)$:
% \begin{subequations}\begin{align}
% \frac{\dot{Z}_\alpha(t)}{Z_\alpha(t)} &= \ddfrac{
% 	\dot{g}(t) + \frac{
% 		e^{-t / \timescale{rise}}
% 	}{
% 		\timescale{rise} (1 - e^{-t / \timescale{rise}})
% 	} g(t)
% }{
% 	g(t)
% }
% \\
% &= \frac{\dot{g}(t)}{g(t)} + \frac{
% 	e^{-t / \timescale{rise}}
% }{
% 	\timescale{rise} (1 - e^{-t / \timescale{rise}})
% }.
% \end{align}\end{subequations}
% Expanding on $\dot{g}(t) / g(t)$:
% \begin{subequations}\begin{align}
% \frac{\dot{g}(t)}{g(t)} &= \left(\frac{g(t)}{\dot{g}(t)}\right)^{-1}
% \\
% &= \left[\ddfrac{
% 	\frac{
% 		\harmonic{dep}{sfh}
% 	}{
% 		\timescale{dep}
% 	} \left(1 - e^{-t / \harmonic{dep}{sfh}}\right) -
% 	\frac{
% 		\hharmonic{dep}{rise}{sfh}
% 	}{
% 		\timescale{dep}
% 	} \left(e^{-t / \timescale{rise}} -
% 	e^{-t / \harmonic{dep}{sfh}}\right)
% }{
% 	\frac{
% 		\hharmonic{dep}{rise}{sfh}
% 	}{
% 		\timescale{rise}\timescale{dep}
% 	}
% 	\left(e^{-t / \timescale{rise}} - e^{-t / \harmonic{dep}{sfh}}
% 	\right)
% }\right]^{-1}
% \\
% &= \frac{1}{\timescale{rise}}\left[
% \frac{
% 	\harmonic{dep}{sfh}
% }{
% 	\hharmonic{dep}{rise}{sfh}
% }\left(
% \frac{
% 	1 - e^{-t / \harmonic{dep}{sfh}}
% }{
% 	e^{-t / \timescale{rise}} - e^{-t / \harmonic{dep}{sfh}}
% }
% \right) - 1\right]^{-1},
% \end{align}\end{subequations}
% and therefore the final expression for~$\dot{Z}_\alpha(t) / Z_\alpha(t)$:
% \begin{equation}
% \frac{\dot{Z}_\alpha(t)}{Z_\alpha(t)} = \frac{
% 	e^{-t / \timescale{rise}}
% }{
% 	\timescale{rise} (1 - e^{-t / \timescale{rise}})
% } + \frac{1}{\timescale{rise}} \left[
% \frac{
% 	\harmonic{dep}{sfh}
% }{
% 	\hharmonic{dep}{rise}{sfh}
% }\left(\frac{
% 	1 - e^{-t / \harmonic{dep}{sfh}}
% }{
% 	e^{-t / \timescale{rise}} - e^{-t / \harmonic{dep}{sfh}}
% }\right) - 1\right]^{-1}
% \end{equation}

