
\documentclass[12pt]{article}
\usepackage[margin = 1in]{geometry}
\usepackage[dvipsnames]{xcolor}
\usepackage{tabularx}
\usepackage{graphicx}
\usepackage{enumitem}
\usepackage{hyperref}
% \usepackage{fancyhdr}
\usepackage[margin = 0cm]{caption}
\usepackage{wrapfig}
\usepackage{amsmath}
\usepackage{amssymb}
\usepackage{natbib}
\setlength{\bibsep}{0pt}
\hypersetup{ % all blue links
	colorlinks		= true,
	urlcolor 		= blue,
	linkcolor 		= blue,
	citecolor 		= blue
}
\setcitestyle{aysep={}, yysep={,}}
\newcommand{\aj}{AJ}
\newcommand{\aap}{A\&A}
\newcommand{\aapr}{A\&ARv}
\newcommand{\apj}{ApJ}
\newcommand{\apjl}{ApJL}
\newcommand{\apjs}{ApJS}
\newcommand{\apss}{Ap\&SS}
\newcommand{\araa}{ARA\&A}
\newcommand{\baas}{BAAS}
\newcommand{\fcp}{Fundamentals Cosmic Phys.}
\newcommand{\mnras}{MNRAS}
\newcommand{\nat}{Nature}
\newcommand{\pasa}{PASA}
\newcommand{\pasp}{PASP}
\newcommand{\ddfrac}[2]{\frac{\displaystyle{#1}}{\displaystyle{#2}}}
\newcommand{\msun}{\ensuremath{\text{M}_\odot}}
\newcommand{\scinote}[2]{\ensuremath{#1\times10^{#2}}}
\providecommand{\noopsort}[1]{}

\newcommand{\ah}{\ensuremath{\text{[$\alpha$/H]}}}
\newcommand{\mh}{\ensuremath{\text{[M/H]}}}
\newcommand{\timescale}[1]{\ensuremath{\tau_\text{#1}}}
\newcommand{\harmonic}[2]{\ensuremath{\bar{\tau}_\text{[#1,#2]}}}
\newcommand{\hharmonic}[3]{\ensuremath{\bar{\tau}_\text{[#1,#2,#3]}}}
\newcommand{\vice}{\textsc{VICE}}
\renewcommand{\tilde}[1]{\ensuremath{\widetilde{#1}}}

\begin{document}

\begin{center}
{\Large \textbf{The Shape and Centroid of Abundance Distributions in One-Zone
Models of Galactic Chemical Evolution}
\par\null\par
James W. Johnson
}
\par\null\par
\rule[0.7\baselineskip]{0.5\textwidth}{0.4pt}
\end{center}

\par\noindent
The motivating question is the origin of the Galactic abundance gradient in the
context of Galactic winds.
Galactic chemical evolution (GCE) models like that of~\citet{Johnson2021}
invoke strong mass loading to explain the observed abundances with yields
predicted by stellar evolution models.
Consequently, each Galactic region quickly reaches some equilibrium abundance
where metal production by young stars is balanced by losses to outflows, and
the gradient arises out of a decrease in the equilibrium abundance with
increasing radius.
However, in models like that of~\citet*{Minchev2013, Minchev2014} and
\citet{Spitoni2019}, the authors neglect mass loading in the outflowing
material, and therefore the equilibrium abundance does not vary with radius.
Both classes of models successfully reproduce the observed gradient, but this
is by construction in the~\citet{Johnson2021} models.
This begs the question as to which parameter choices must be made in order to
successfully reproduce this empirical result in the latter class of models.

\par\null\par\noindent
{\large \textbf{Generic MDF Optimization Criteria}}
\par\noindent
To this end, here I derive analytic and parametric expressions for the
metallicity distribution function (MDF) given a set of choices of GCE model
parameters.
The MDF in the metal mass fraction~$Z$ can be expressed as:
\begin{equation}
\frac{dN}{dZ} = \frac{dN}{dt} \bigg/ \frac{dZ}{dt}
\propto \frac{\dot{M}_\star}{\dot{Z}},
\label{eq:dndz}
\end{equation}
where~$\dot{M}_\star$ is the star formation rate (SFR).
In general,~$\dot{M}_\star$ and~$\dot{Z}$ will be parametric functions of time.
As a consequence of the abundance gradient, the mode of the MDF shifts from
metal-rich to metal-poor with increasing radius
\citep[see, e.g.,][]{Hayden2015}.
Therefore, it is also interesting to estimate the position of the maximum in
the MDF by taking its derivative and setting it equal to zero:
\begin{subequations}\begin{align}
\frac{d^2N}{dZ^2} &= \frac{d}{dZ} \left(\frac{dN}{dZ}\right)
\\
&= \frac{1}{\dot{Z}} \frac{d}{dt} \left(\frac{dN}{dZ}\right)
\\
&\propto \frac{1}{\dot{Z}} \frac{d}{dt}
\left(\frac{\dot{M}_\star}{\dot{Z}}\right)
\end{align}
\\
\begin{align}
&= \frac{1}{\dot{Z}} \left(\frac{
	\dot{Z}\ddot{M}_\star - \ddot{Z}\dot{M}_\star
}{
	\dot{Z}^2
}\right)
\\
&= \frac{\dot{M}_\star}{\dot{Z}^2}
\left(\frac{\ddot{M}_\star}{\dot{M}_\star} - \frac{\ddot{Z}}{\dot{Z}}\right).
\label{eq:d2ndz2}
\end{align}\end{subequations}
By setting the above expression equal to zero, it is apparent that maxima in the
MDF in~$Z$ occurs when
\begin{equation}
\frac{\ddot{M}_\star}{\dot{M}_\star} = \frac{\ddot{Z}}{\dot{Z}}.
\label{eq:zmdf_maxima_criterion}
\end{equation}
\par
In practice, however, abundances are quantified logarithmically.
It is therefore also interesting to compute the MDF in~\mh,\footnote{
	I adopt standard notation where~$\mh \equiv \log_{10}(Z / Z_/\odot) -
	\log_{10}(X / X_\odot)$ where~$Z_\odot$ is the metal mass fraction of the
	sun and~$X$ and~$X_\odot$ are the corresponding hydrogen mass fractions.
	However, since hydrogen mass fractions do not vary much (especially when
	scalled logarithmically), here I assume that~$\log_{10}(X / X_\odot)
	\approx 0$ and therefore~$\mh \approx \log_{10}(Z / Z_\odot)$.
} which is straight-forward given the derivations already written out above.
The MDF in~\mh~itself:
\begin{equation}
\frac{dN}{d\mh} = \ln 10 \left(\frac{Z}{Z_\odot}\right) \frac{dN}{dZ}
\propto \frac{\dot{M}_\star}{\dot{Z} / Z},
\label{eq:dndmh}
\end{equation}
and its derivative:
\begin{subequations}\begin{align}
\frac{d^2N}{d\mh^2} &= \frac{d}{d\mh} \left(\frac{dN}{d\mh}\right)
\\
&= \ln 10 \left(\frac{Z}{Z_\odot}\right) \frac{d}{dZ}
\left(\ln 10 \frac{Z}{Z_\odot} \frac{dN}{dZ}\right)
\\
&= \left(\frac{\ln 10}{Z_\odot}\right)^2 Z\frac{d}{dZ}
\left(Z\frac{dN}{dZ}\right)
\\
&= \left(\frac{\ln 10}{Z_\odot}\right)^2
Z\left(\frac{dN}{dZ} + Z\frac{d^2N}{dZ^2}\right)
\\
&= \left(\frac{\ln 10}{Z_\odot}\right)^2 \left(Z\frac{dN}{dZ} +
Z^2 \frac{d^2N}{dZ^2}\right)
\\
&\propto \frac{\dot{M}_\star}{\dot{Z} / Z} +
\frac{\dot{M}_\star}{(\dot{Z} / Z)^2}
\left(\frac{\ddot{M}_\star}{\dot{M}_\star} - \frac{\ddot{Z}}{\dot{Z}}\right).
\label{eq:d2ndmh2}
\end{align}\end{subequations}
By setting equation~\ref{eq:d2ndmh2} equal to zero, it then follows that maxima
in the MDF in~\mh~occur when
\begin{equation}
\frac{\ddot{M}_\star}{\dot{M}_\star} = \frac{\ddot{Z}}{\dot{Z}} -
\frac{\dot{Z}}{Z}.
\label{eq:mhmdf_maxima_criterion}
\end{equation}
Solving this equation will first yield a time at which the maximum is produced,
which can then be converted into metallicity by evaluating~$Z(t)$ at the
derived time.

\par\null\par\noindent
{\large \textbf{Enrichment Rates}}
\par\noindent
In the interest of analytic/parametric solutions, I assume throughout this
document that the star formation efficiency (SFE) timescale
$\tau_\star \equiv M_\text{gas} / \dot{M}_\star$ and the outflow mass loading
factor~$\eta \equiv \dot{M}_\text{out} / \dot{M}_\star$ are both fixed in time.
In the intereset of simplicity, I also focus on alpha elements with negligible
yields from asymptotic giant branch stars and Type Ia supernovae (e.g., O and
Mg), and I adjust my notation accordingly from~$Z$ to~$Z_\alpha$
and~\mh~to~\ah~hereafter.
\par
From~\citet*{Weinberg2017}, the enrichment rate of alpha elements can be
expressed as
\begin{equation}
\dot{M}_\alpha = y_\alpha \dot{M}_\star - Z_\alpha \dot{M}_\star (1 + \eta - r),
\label{eq:mdot_alpha}
\end{equation}
where~$y_\alpha$ is a population-averaged alpha element yield from massive
stars, assumed to be ejected immediately after a stellar population forms.
This assumption is valid due to the short lifetimes of massive stars compared
to the relevant evolutionary timescales of the Milky Way.
$r$ is a term which accounts for the return of stellar envelopes back to
the ISM ($r \approx 0.4$ for a~\citealt{Kroupa2001} IMF;
\citealp*{Weinberg2017}).
\par
Based on the definition of~$Z_\alpha$, the quotient rule implies that its
time-derivative can be expressed as
\begin{subequations}\begin{align}
\dot{Z}_\alpha &= \frac{
	M_\text{g} \dot{M}_\alpha - M_\alpha \dot{M}_\text{g}
}{
	M_\text{g}^2
}
\\
&= \frac{
	M_\text{g} (y_\alpha \dot{M}_\star - Z_\alpha \dot{M}_\star (1 + \eta - r))
	- M_\alpha \ddot{M}_\star \tau_\star
}{
	M_\text{g}^2
}
\\
&= y_\alpha \frac{\dot{M}_\star}{M_\text{g}} -
Z_\alpha \frac{\dot{M}_\star}{M_\text{g}}(1 + \eta - r) -
Z_\alpha \frac{\ddot{M}_\star}{M_\text{g}} \tau_\star
\\
&= \frac{y_\alpha}{\tau_\star} -
Z_\alpha \left(\frac{1 + \eta - r}{\tau_\star} +
\frac{\ddot{M}_\star}{\dot{M}_\star}\right),
\label{eq:dzdt}
\end{align}\end{subequations}
which yields the following linear ODE for~$Z_\alpha(t)$:
\begin{equation}
\dot{Z}_\alpha + Z_\alpha\left(\frac{1}{\timescale{dep}} +
\frac{\ddot{M}_\star}{\dot{M}_\star}\right) = \frac{y_\alpha}{\tau_\star},
\label{eq:generic-ode}
\end{equation}
where I have substituted in for the depletion time~$\timescale{dep} \equiv
\tau_\star / (1 + \eta - r)$.
The depletion time quantifies the e-folding timescale on which the interstellar
medium (ISM) gas would be deplete due to star formation and outflows in the
absence of gas accretion.
With accretion, however, it describes a timescale on which the ISM fully
recycles; that is, a single fluid element will be either incorporated into new
stars or ejected in an outflow on average within one depletion time.
\par
Equation~\ref{eq:dzdt} can also be used to derive the equilibrium abundance,
at which metal production is balanced by depletion onto new stars and losses to
outflows.
$\dot{Z}_\alpha = 0$ at the equilibrium abundance by definition, and solving
for~$Z_\alpha$ is trivial:
\begin{equation}
Z_{\alpha,\text{eq}} = \ddfrac{y_\alpha}{
	1 + \eta - r + \tau_\star \frac{\ddot{M}_\star}{\dot{M}_\star}
}.
\end{equation}
\par
Denoting~$f(t)$ as the multiplicative prefactor on~$Z_\alpha$ in equation
\ref{eq:generic-ode}, which may or may not vary with time for a given
parameterization of the SFH, the solution to~$Z(t)$ can be expressed as:
\begin{subequations}\begin{align}
Z_\alpha(t) &= \left(
\exp\left(-\int f(t') dt'\right) \left(\int_0^t \exp\left(\int f(t') dt'\right)
\frac{y_\alpha}{\tau_\star}dt' + C\right)
\right)
\label{eq:generic-ode-solution}
\\
f(t) &= \frac{1}{\timescale{dep}} + \frac{\ddot{M}_\star}{\dot{M}_\star},
\label{eq:f}
\end{align}\end{subequations}
where the integration constant~$C$ should be assigned such that the initial
condition~$Z_\alpha(t = 0) = 0$ is satisfied.
The generic procedure for optimizing the MDF given a choice of GCE parameters
is as follows:
\begin{enumerate}

	\item Differentiate the SFH with time to obtain and expression
	for~$\ddot{M}_\star / \dot{M}_\star$.

	\item Plug~$\ddot{M}_\star / \dot{M}_\star$ into equation~\ref{eq:f} and
	solve equation~\ref{eq:generic-ode-solution} for~$Z_\alpha(t)$.
	Ensure that the integration constant~$C$ is chosen such
	that~$Z_\alpha(t = 0) = 0$.

	\item Differentiate~$Z_\alpha(t)$ with time to obtain an expression
	for~$\dot{Z}_\alpha(t)$.
	If optimizing the distribution in~\ah, differentiate with time once more
	and additionally obtain the expression
	for~$\ddot{Z}_\alpha(t) / \dot{Z}_\alpha(t)$.

	\item Solve equation~\ref{eq:zmdf_maxima_criterion} and/or
	equation~\ref{eq:mhmdf_maxima_criterion} to obtain the time at which the
	maximum in the MDF occurs.
	Depending on the nature of the expression, this solution may be numerical
	as opposed to analytic.

	\item Plug the time into the previously derived expression for~$Z_\alpha(t)$
	and compute~\ah~if optimizing the distribution in~\ah.

\end{enumerate}

\par\null\par\noindent
{\large \textbf{Exponential SFHs}}
\par\noindent
Here I derive the solution to mode(\ah) for an exponential SFH.
The solution for a constant SFH then follows by letting the e-folding timescale
of the SFH~$\timescale{sfh} \rightarrow \infty$.
The first step is to differentiate the SFH with time; taking~$A$ to denote
the overall normalization:
\begin{subequations}\begin{align}
\ddot{M}_\star &= \frac{d}{dt} A e^{-t / \timescale{sfh}}
\\
&= \frac{-A}{\timescale{sfh}} e^{-t / \timescale{sfh}}
\\
\implies \frac{\ddot{M}_\star}{\dot{M}_\star} &= \frac{-1}{\timescale{sfh}}.
\end{align}\end{subequations}
As previously shown in~\citet*{Weinberg2017}, it then follows that the
equilibrium abundance for an exponential SFH is given by
\begin{equation}
Z_{\alpha,\text{eq}} = \frac{y_\alpha}{
	1 + \eta - r - \tau_\star / \timescale{sfh}
}.
\label{eq:zeq-expsfh}
\end{equation}
\par
At this point, it is helpful to adopt the harmonic timescale notation from
\citet*{Weinberg2017}
\begin{equation}
\harmonic{X}{Y} \equiv \left(\frac{1}{\timescale{X}} -
\frac{1}{\timescale{Y}}\right)^{-1} = \frac{
	\timescale{Y}\timescale{X}
}{
	\timescale{Y} - \timescale{X}
}
\end{equation}
which arises when integrating exponentials depending on multiple timescales.
$f(t)$ for an exponential SFH can then be denoted as~$f(t) = 1 /
\harmonic{dep}{sfh}$, and equation~\ref{eq:generic-ode} then becomes:
\begin{subequations}\begin{align}
Z_\alpha(t) &= \exp\left(-\int \frac{1}{\harmonic{dep}{sfh}}dt'\right)
\left(
\int_0^t \exp\left(\int \frac{1}{\harmonic{dep}{sfh}}dt'\right)
\frac{y_\alpha}{\tau_\star} dt' + C
\right)
% intermediate steps written out here but commented out.
% \\
% &= \frac{y_\alpha}{\tau_\star} e^{-t / \harmonic{dep}{sfh}} \left(
% \int_0^t e^{t' / \harmonic{dep}{sfh}} dt' + C \right)
% \\
% &= \frac{y_\alpha}{\tau_\star} e^{-t / \harmonic{dep}{sfh}} \left(
% \harmonic{dep}{sfh} e^{t' / \harmonic{dep}{sfh}} \Bigg|_0^t + C
% \right)
% \\
% &= \frac{y_\alpha}{\tau_\star} e^{-t / \harmonic{dep}{sfh}}
% \left(
% \harmonic{dep}{sfh} \left(e^{t / \harmonic{dep}{sfh}} - 1\right) + C\right)
% \\
% &= y_\alpha \frac{\harmonic{dep}{sfh}}{\tau_\star} \left(1 -
% e^{-t / \harmonic{dep}{sfh}} + Ce^{-t / \harmonic{dep}{sfh}}\right)
% \\
% &= y_\alpha \frac{\timescale{sfh}\timescale{dep}}{
% 	\tau_\star\left(\timescale{sfh} - \timescale{dep}\right)
% } \left(1 - e^{-t / \harmonic{dep}{sfh}} + Ce^{-t / \harmonic{dep}{sfh}}\right)
\\
&= \frac{y_\alpha}{1 + \eta - r - \tau_\star / \timescale{sfh}}
\left(1 - e^{-t / \harmonic{dep}{sfh}}\right)
\\
&= Z_{\alpha,\text{eq}}\left(1 - e^{-t / \harmonic{dep}{sfh}}\right).
\end{align}\end{subequations}
In this case, the integration constant~$C = 0$.
The first and second time-derivatives of~$Z_\alpha(t)$ then follow trivially:
\begin{subequations}\begin{align}
\dot{Z}_\alpha(t) &= \frac{Z_{\alpha, \text{eq}}}{\harmonic{dep}{sfh}}
e^{-t / \harmonic{dep}{sfh}}
\\
\ddot{Z}_\alpha(t) &= \frac{-Z_{\alpha, \text{eq}}}{\harmonic{dep}{sfh}^2}
e^{-t / \harmonic{dep}{sfh}}.
\end{align}\end{subequations}
Plugging these expressions into the~$Z_\alpha$ MDF optimization criterion
(equation~\ref{eq:zmdf_maxima_criterion}) results in the following expression:
\begin{equation}
\frac{1}{\timescale{sfh}} = \frac{1}{\harmonic{dep}{sfh}},
\end{equation}
an expression which does not depend on time.
This form of SFH therefore cannot produce a maximum in~$dN / dZ_\alpha$, though
I show below that the opposite is the case for~$dN / d\ah$.
\par
Plugging these expressions into the~\ah~MDF optimization criterion (equation
\ref{eq:mhmdf_maxima_criterion}) then yields the following expression for the
time at which the maximum in the MDF occurs:
\begin{subequations}\begin{align}
\frac{1}{\timescale{sfh}} &= \frac{1}{\harmonic{dep}{sfh}} +
\frac{1}{\harmonic{dep}{sfh}} \left(\frac{
	e^{-t_\text{max} / \harmonic{dep}{sfh}}
}{
	1 - e^{-t_\text{max} / \harmonic{dep}{sfh}}
}\right)
\\
% intermediate steps to the solution simply commented out here
% \implies \frac{\harmonic{dep}{sfh}}{\timescale{sfh}} &= 1 + \frac{
% 	e^{-t / \harmonic{dep}{sfh}}
% }{
% 	1 - e^{-t / \harmonic{dep}{sfh}}
% }
% \\
% &= \frac{
% 	1 - e^{-t / \harmonic{dep}{sfh}} + e^{-t / \harmonic{dep}{sfh}}
% }{
% 	1 - e^{-t / \harmonic{dep}{sfh}}
% }
% \\
% &= \frac{1}{1 - e^{-t / \harmonic{dep}{sfh}}}
% \\
% \implies 1 - e^{-t / \harmonic{dep}{sfh}} &= \frac{\timescale{sfh}}{
% 	\harmonic{dep}{sfh}
% } 
% \\
% &= \frac{\timescale{sfh}}{\timescale{dep}} - 1
% \\
% \implies e^{-t / \harmonic{dep}{sfh}} &= 2 - \frac{\timescale{sfh}}{
% 	\timescale{dep}
% }
% \\
\implies t_\text{max} &= -\harmonic{dep}{sfh} \ln \left( 2 -
\frac{\timescale{sfh}}{\timescale{dep}}\right).
\end{align}\end{subequations}
Evaluation~$Z_\alpha(t)$ at this time yields
\begin{subequations}\begin{align}
Z_\alpha(t_\text{max}) &= Z_{\alpha,\text{eq}} \left(
1 - e^{\ln \left(2 - \timescale{sfh} / \timescale{dep}\right)}
\right)
% \\
% &= Z_{\alpha,\text{eq}} \left(1 - 2 + \frac{\timescale{sfh}}{\timescale{dep}}
% \right)
\\
&= Z_{\alpha,\text{eq}} \left(\frac{\timescale{sfh}}{\timescale{dep}} -
1\right).
\end{align}\end{subequations}
At this point we can assert that a true turnover in the MDF will occur if and
only if~$Z(t_\text{max}) < Z_{\alpha,\text{eq}}$, because the ISM will never
reach a super-equilibrium abundance given this smooth SFH.
If~$Z(t_\text{max}) > Z_{\alpha,\text{eq}}$, then the mode of the~\ah~MDF
instead coincides with a sharp cutoff at the metallicity of the present day,
which may be arbitrarily close to or significantly below the equilibrium
abundance.
The assertion that~$Z(t_\text{max}) < Z_{\alpha,\text{eq}}$ leads to the
following condition which, if satisfied, means a true turnover in the~\ah~MDF
will be produced for a given choice of GCE parameters:
\begin{subequations}\begin{align}
\frac{\timescale{sfh}}{\timescale{dep}} - 1 &< 1
\\
\implies \timescale{sfh} &< 2\timescale{dep}.
\label{eq:mode-ah-expsfh-criterion}
\end{align}\end{subequations}
It may be tempting to also assert that~$Z_{\alpha,\text{eq}} > 0$, though close
inspection of equation~\ref{eq:zeq-expsfh} indicates
that~$Z_{\alpha,\text{eq}} < 0$ simply corresponds to cases in
which~$\timescale{sfh} < \timescale{dep}$.
In these cases, the prefactor~$\timescale{sfh} / \timescale{dep} - 1$ is also
negative, and therefore the position of the peak is positive.
\par
Finally, solving for the position of the peak in the~\ah~MDF yields
\begin{equation}
\ah_\text{max} = \log_{10}\left[
\frac{Z_{\alpha,\text{eq}}}{Z_{\alpha,\odot}}
\left(\frac{\timescale{sfh}}{\timescale{dep}} - 1\right)
\right].
\label{eq:mode-ah-expsfh}
\end{equation}
When~$\timescale{sfh} = \timescale{dep}$,~$\ah_\text{max}$ reaches
an~$\infty \times 0$ indeterminate form.
I find the appropriate expression for mode(\ah) in this instance by taking
the limit:
\begin{subequations}\begin{align}
\lim_{\timescale{sfh} \rightarrow \timescale{dep}} \ah_\text{max}
&= \lim_{\timescale{sfh} \rightarrow \timescale{dep}} \log_{10}
\left[\frac{Z_{\alpha,\text{eq}}}{Z_{\alpha,\odot}}
\left(\frac{\timescale{sfh}}{\timescale{dep}} - 1\right)\right]
\\
&= \log_{10} \lim_{\timescale{sfh} \rightarrow \timescale{dep}} \left[
\frac{y_\alpha}{
	Z_{\alpha,\odot} \left(1 + \eta - r - \tau_\star / \timescale{sfh}\right)
}
\left(\frac{\timescale{sfh} - \timescale{dep}}{\timescale{dep}}\right)
\right]
\\
% &= \log_{10} \lim_{\timescale{sfh} \rightarrow \timescale{dep}} \left[
% \frac{y_\alpha}{
% 	Z_{\alpha,\odot}\left(1 + \eta - r\right)
% 	\left(1 - \timescale{dep} / \timescale{sfh}\right)
% }
% \left(\frac{\timescale{sfh} - \timescale{dep}}{\timescale{dep}}\right)
% \right]
% \\
% &= \log_{10} \left[
% \frac{y_\alpha}{
% 	Z_{\alpha,\odot}\left(1 + \eta - r\right)
% }
% \lim_{\timescale{sfh} \rightarrow \timescale{dep}}
% \left(
% \frac{
% 	\timescale{sfh} - \timescale{dep}
% }{
% 	\timescale{dep} - \timescale{dep}^2 / \timescale{sfh}
% }
% \right)
% \right]
% \\
% &= \log_{10} \left[
% \frac{y_\alpha}{
% 	Z_{\alpha,\odot}(1 + \eta - r)
% }
% \lim_{\timescale{sfh} \rightarrow \timescale{dep}}
% \left(
% \frac{1}{
% 	\timescale{dep}^2 / \timescale{sfh}^2
% }
% \right)
% \right] (\text{LH})
% \\
&= \log_{10} \left[
\frac{y_\alpha}{
	Z_{\alpha,\odot} \left(1 + \eta - r\right)
}
\right],
\end{align}\end{subequations}
which is, perhaps coincidentally, equivalent to the equilibrium~\ah~abundance
for a constant SFH.
We therefore arrive at the full piece-wise solution for mode(\ah):
\begin{equation}
\text{mode}(\ah) = \begin{cases}
\ah(t = t_\text{final}) & (\timescale{sfh} > 2\timescale{dep}~\text{or}~
t_\text{max} > 13.2~\text{Gyr})
\\
\null
\\
\log_{10} \left[
\ddfrac{y_\alpha}{
	Z_{\alpha,\odot} \left(1 + \eta - r\right)
}
\right] & (\timescale{sfh} = \timescale{dep})
\\
\null
\\
\log_{10}\left[
\ddfrac{Z_{\alpha,\text{eq}}}{Z_{\alpha,\odot}}
\left(\ddfrac{\timescale{sfh}}{\timescale{dep}} - 1\right)
\right] & (\text{otherwise}).
\end{cases}
\label{eq:mode-ah-expsfh-piecewise}
\end{equation}

\begin{figure}
\centering
\includegraphics[scale = 0.55]{mode_oh.pdf}
\includegraphics[scale = 0.55]{mode_oh_versus_tausfh.pdf}
\caption{
	The dependence of mode(\ah) on the e-folding timescale of an exponential
	SFH~\timescale{sfh}.
	\textbf{Left}:~\ah~distributions and the associated mode for a handful of
	choices of~$\timescale{sfh}$ for a given~$\timescale{dep}$.
	For these models I adopt~$\tau_\star$ = 2 Gyr and~$\eta = 0$ with
	yields and solar abundances appropriate for oxygen ($y_\alpha = 0.005$,
	$Z_{\alpha,\odot} = 0.00572$ based on~\citealt{Asplund2009}).
	E-folding timescales of the SFH are chosen such that they are a
	round-number multiple of the depletion time, color-coded according to the
	legend.
	We mark the position of mode(\ah) computed according to equation
	\ref{eq:mode-ah-expsfh-piecewise} in solid lines, while the~\ah~MDFs
	computed with~\vice~are shown in the dotted lines.
	\textbf{Right}: mode(\ah) as a function of~\timescale{sfh} for a handful of
	choices of~$\tau_\star$ and~$\eta$.
}
\label{fig:mode-ah-expsfh}
\end{figure}

\par\null\par\noindent
{\large \textbf{Constant SFHs}}
\par\noindent
The solutions for a constant SFH can be derived from the soluations for an
exponential SFH by simply taking the limit
as~$\timescale{sfh} \rightarrow \infty$.
Interestingly, this violates the condition that~$\timescale{sfh} <
2\timescale{dep}$, indicating immediately that this form of an SFH will always
produce an MDF with a sharp cutoff as opposed to a true turnover.
No further calculations are necessary.

\par\null\par\noindent
{\large \textbf{Linear-Exponential SFHs}}
\par\noindent
The solution for a linear-exponential SFH~$\dot{M}_\star \propto t e^{-t /
\timescale{sfh}}$ is somewhat more complicated.
Denoting the normalization of the SFH as~$A$:
\begin{subequations}\begin{align}
\ddot{M}_\star &= \frac{d}{dt} Ate^{-t / \timescale{sfh}}
\\
&= Ae^{-t / \timescale{sfh}} - A\frac{t}{\timescale{sfh}}
e^{-t / \timescale{sfh}}
\\
&= Ate^{-t / \timescale{sfh}} \left(\frac{1}{t} - \frac{1}{\timescale{sfh}}
\right)
\\
\implies \frac{\ddot{M}_\star}{\dot{M}_\star} &= \frac{1}{t} -
\frac{1}{\timescale{sfh}}
\\
\implies f(t) &= \frac{1}{\timescale{dep}} +
\frac{\ddot{M}_\star}{\dot{M}_\star} = \frac{1}{t} +
\frac{1}{\harmonic{dep}{sfh}},
\end{align}\end{subequations}
and now solving for the integral of~$f(t)$
\begin{subequations}\begin{align}
\int f(t) dt &= \int \left(\frac{1}{t} + \frac{1}{\harmonic{dep}{sfh}}\right) dt
\\
&= \ln t + \frac{t}{\harmonic{dep}{sfh}}
\end{align}\end{subequations}
yields the following expression for~$Z_\alpha(t)$:
\begin{subequations}\begin{align}
Z_\alpha(t) &= \frac{1}{t}e^{-t / \harmonic{dep}{sfh}} \left[
\frac{y_\alpha}{\tau_\star} \int_0^t t' e^{t' / \harmonic{dep}{sfh}} dt' + C
\right]
\\
&= \frac{1}{t}e^{-t / \harmonic{dep}{sfh}} \left[
\frac{y_\alpha}{\tau_\star}\left(
\harmonic{dep}{sfh}^2 + e^{t / \harmonic{dep}{sfh}} \left(\harmonic{dep}{sfh}
t - \harmonic{dep}{sfh}^2\right)
\right) + C\right]
\\
&= \frac{y_\alpha}{\tau_\star} \left[
\harmonic{dep}{sfh} + \frac{\harmonic{dep}{sfh}^2}{t}
\left(e^{-t / \harmonic{dep}{sfh}} - 1\right)
\right],
\end{align}\end{subequations}
where I have substituted~$C = 0$ into the final equality to satisfy the
boundary condition that~$Z_\alpha(t = 0) = 0$.
The first and second time-derivatives then follow:
\begin{subequations}\begin{align}
\dot{Z}_\alpha(t) &= \frac{y_\alpha}{\tau_\star} \left[
\frac{-\harmonic{dep}{sfh}^2}{t^2} \left(e^{-t / \harmonic{dep}{sfh}} - 1\right)
- \frac{\harmonic{dep}{sfh}}{t} e^{-t / \harmonic{dep}{sfh}}
\right]
\\
\ddot{Z}_\alpha(t) &= \frac{y_\alpha}{\tau_\star} \left[
\frac{2\harmonic{dep}{sfh}^2}{t^3} \left(e^{-t / \harmonic{dep}{sfh}} - 1\right)
+ e^{-t / \harmonic{dep}{sfh}} \left(\frac{2\harmonic{dep}{sfh}}{t^2} +
\frac{1}{t}\right)
\right]
\end{align}\end{subequations}














% \par
% For the remainder of this document, I derive solutions to the above expressions
% for the ``inside-out'' star formation history (SFH) from~\citet{Johnson2021}:
% \begin{equation}
% \dot{M}_\star(t) \propto \left(1 - e^{-t / \timescale{rise}}\right)
% e^{-t / \timescale{sfh}}.
% \label{eq:insideout_sfh}
% \end{equation}
% The physically interesting advantage of this parameterization over the classic
% ``linear-times-exponential''~$t e^{-t / \timescale{sfh}}$ form is that the
% rise timescale~\timescale{rise}~gives one some control over how long the SFH is
% increasing independent of~\timescale{sfh}.
% In the interest of analytic/parametric solutions, I assume that the star
% formation efficiency (SFE) timescale~$\tau_\star \equiv M_\text{g} /
% \dot{M}_\star$ and the outflow mass loading factor~$\eta \equiv
% \dot{M}_\text{out} / \dot{M}_\star$ are both constant in time.
% In the interest of simplicity, I also focus on alpha elements with negligible
% yields from Type Ia supernovae (e.g., O and Mg), and I adjust my notation
% accordingly from~$Z$ to~$Z_\alpha$ and~\mh~to~\ah~hereafter.

\par\null\par\noindent
\textbf{Analytic solution to~$Z_\alpha(t)$}
\par\noindent
The expression for~$\dot{M}_\star$ is given by the assumption of equation
\ref{eq:insideout_sfh}, so now I must solve for~$Z_\alpha(t)$ using arguments
similar to those of~\citet*{Weinberg2017}.
For alpha elements,~\citet*{Weinberg2017} express the rate of change of the
mass present in the ISM as
% \begin{equation}
% \dot{M}_\alpha = y_\alpha \dot{M}_\star - Z_\alpha \dot{M}_\star (1 + \eta - r),
% \label{eq:mdot_alpha}
% \end{equation}
% where~$y_\alpha$ is a population-averaged alpha element yield from massive
% stars, assumed to be ejected immediately after a stellar population forms.
% This assumption is valid due to the short lifetimes of massive stars compared
% to the relevant evolutionary timescales of the Milky Way.
% $r$ is a term which accounts for the return of stellar envelopes back to
% the ISM ($r \approx 0.4$ for a~\citealt{Kroupa2001} IMF;
% \citealp*{Weinberg2017}).
\par
Based on the definition of~$Z_\alpha$, the quotient rule implies that its
time-derivative can be expressed as
\begin{subequations}\begin{align}
\dot{Z}_\alpha &= \frac{
	M_\text{g} \dot{M}_\alpha - M_\alpha \dot{M}_\text{g}
}{
	M_\text{g}^2
}
\\
&= \frac{
	M_\text{g} (y_\alpha \dot{M}_\star - Z_\alpha \dot{M}_\star (1 + \eta - r))
	- M_\alpha \ddot{M}_\star \tau_\star
}{
	M_\text{g}^2
}
\\
&= y_\alpha \frac{\dot{M}_\star}{M_\text{g}} -
Z_\alpha \frac{\dot{M}_\star}{M_\text{g}}(1 + \eta - r) -
Z_\alpha \frac{\ddot{M}_\star}{M_\text{g}} \tau_\star
\\
&= \frac{y_\alpha}{\tau_\star} -
Z_\alpha \left(\frac{1 + \eta - r}{\tau_\star} +
\frac{\ddot{M}_\star}{\dot{M}_\star}\right).
% \label{eq:dzdt}
\end{align}\end{subequations}
The next step is to differentiate the SFH with time to compute
$\ddot{M}_\star / \dot{M}_\star$. Taking~$A$ to denote the overall normalizing
factor:
\begin{subequations}\begin{align}
\ddot{M}_\star &= \frac{d}{dt}
A(1 - e^{-t / \timescale{rise}}) e^{-t / \timescale{sfh}}
\\
&= A \frac{1}{\timescale{rise}}e^{-t / \timescale{rise}}
e^{-t / \timescale{sfh}} +
A(1 - e^{-t / \timescale{rise}}) \frac{-1}{\timescale{sfh}}
e^{-t / \timescale{sfh}}
\\
&= A(1 - e^{-t / \timescale{rise}})e^{-t / \timescale{sfh}}
\left(\frac{
	e^{-t / \timescale{rise}}
}{
	\timescale{rise}(1 - e^{-t / \timescale{rise}})
} - \frac{1}{\timescale{sfh}}
\right)
\\
\implies \frac{\ddot{M}_\star}{\dot{M}_\star} &= \frac{
	e^{-t / \timescale{rise}}
}{
	\timescale{rise}(1 - e^{-t / \timescale{rise}})
} - \frac{1}{\timescale{sfh}}.
\label{eq:mddotstar-over-mdotstar}
\end{align}\end{subequations}
At this point, it is straightforward to compute the equilibrium alpha element
abundance for this SFH by plugging equation~\ref{eq:mddotstar-over-mdotstar}
into equation~\ref{eq:dzdt} and setting~$\dot{Z}_\alpha = 0$.
This procedure results in the following expression:
\begin{equation}
Z_{\alpha,\text{eq}} = \ddfrac{
	y_\alpha
}{
	1 + \eta - r + \frac{
		\tau_\star e^{-t / \timescale{rise}}
	}{
		\timescale{rise}(1 - e^{-t / \timescale{rise}})
	} - \frac{\tau_\star}{\timescale{sfh}}
}.
\label{eq:zalpha-eq}
\end{equation}
Equation~\ref{eq:zalpha-eq} indicates that unlike the simpler SFHs like a
constant or single exponential, the equilibrium abundance is not uniform in
time.
Instead, the term in the denominator depending on~\timescale{rise} is infinite
at~$t = 0$.
Therefore, the equilibrium abundance is zero at~$t = 0$, increases at early
times, and then approaches the solution for a single exponential
once~$t \gg \timescale{rise}$.
As a sanity check, equation~\ref{eq:zalpha-eq} indeed reduces to the
equilibrium abundance for a single exponential SFH found
by~\citet*{Weinberg2017} if I let~$\timescale{rise} \rightarrow 0$.
\par
Plugging this into equation~\ref{eq:dzdt} above yields the following linear
ODE for~$Z_\alpha$:
\begin{equation}
\dot{Z}_\alpha + Z_\alpha \left(\frac{1}{\timescale{dep}} +
\frac{
	e^{-t / \timescale{rise}}
}{
	\timescale{rise}(1 - e^{-t / \timescale{rise}})
} - \frac{1}{\timescale{sfh}}\right) = \frac{y_\alpha}{\tau_\star},
\end{equation}
where I have substituted for the depletion time, defined as
$\timescale{dep} \equiv \tau_\star / (1 + \eta - r)$.
$\timescale{dep}$ quantifies the e-folding timescale on which the ISM would
decline due to both star formation and mass loading, if present.
For notational convenience, I define the function~$f(t)$ to denote the
multiplicative factor on~$Z_\alpha$:
\begin{equation}
f(t) \equiv \frac{1}{\timescale{dep}} + \frac{
	e^{-t / \timescale{rise}}
}{
	\timescale{rise}(1 - e^{-t / \timescale{rise}})
} - \frac{1}{\timescale{sfh}},
\end{equation}
such that the general solution for~$Z_\alpha$ can be expressed as
\begin{equation}
Z_\alpha(t) = \exp\left(-\int f(t') dt'\right)\left(
\int_0^t \exp\left(\int f(t') dt'\right) \frac{y_\alpha}{\tau_\star} dt' + C
\right),
\label{eq:za-linear-ode}
\end{equation}
where the constant~$C$ will later be assigned such that the initial
condition~$Z_\alpha(t = 0) = 0$ is satisfied.
At this point, it is helpful to zoom in on the integral of~$f(t)$.
Its term depending on~$\timescale{rise}$ can be integrated with a series of
variable substitutions:
\begin{subequations}\begin{align}
\int \frac{
	e^{-t / \timescale{rise}}
}{
	\timescale{rise}(1 - e^{-t / \timescale{rise}})
} dt
&= \int \frac{e^{-u}}{1 - e^{-u}} du
\qquad \left(u = \frac{t}{\timescale{rise}};~~du = \frac{1}{\timescale{rise}}
dt\right)
\\
&= \int \frac{dv}{v} \qquad \left(v = 1 - e^{-u};~~dv = e^{-u} du\right)
\\
&= \ln v
\\
&= \ln (1 - e^{-u})
\\
&= \ln (1 - e^{-t / \timescale{rise}})
\\
\implies \int f(t) dt
&= \frac{t}{\timescale{dep}} + \ln (1 - e^{-t / \timescale{rise}}) -
\frac{t}{\timescale{sfh}}.
\end{align}\end{subequations}
Solving equation~\ref{eq:za-linear-ode} amounts to integrating the sum of a
handful of exponentials with prefactors that depend on~\timescale{rise},
\timescale{dep}, and~\timescale{sfh} in a non-trivial way.
The full derivation is attached, and the final solution is given by
\begin{equation}
\begin{split}
Z_\alpha(t) &= \frac{1}{1 - e^{-t / \timescale{rise}}}
\left(\frac{y_\alpha}{1 + \eta - r}\right)
\bigg[\frac{
	\timescale{sfh}
}{
	\timescale{sfh} - \timescale{dep}
} \left(
1 - \exp\left(-t\frac{
	\timescale{sfh} - \timescale{dep}
}{
	\timescale{sfh}\timescale{dep}
}\right)
\right) -
\\
&\qquad \frac{
	\timescale{sfh}\timescale{rise}
}{
	\timescale{sfh}\timescale{rise} - \timescale{dep}\timescale{rise} -
	\timescale{sfh}\timescale{dep}
} \left(e^{-t / \timescale{rise}} -
\exp\left(-t
\frac{
	\timescale{sfh} - \timescale{dep}
}{
	\timescale{sfh}\timescale{dep}
}
\right)
\right)\bigg].
\end{split}
\end{equation}
Adopting the harmonic timescale notation~$\harmonic{X}{Y} = \timescale{Y}
\timescale{X} / (\timescale{Y} - \timescale{X})$ of~\citet*{Weinberg2017}
simplifies the expression:
\begin{equation}
\begin{split}
Z_\alpha(t) &= \frac{1}{1 - e^{-t / \timescale{rise}}} \left( \frac{
	y_\alpha
}{
	1 + \eta - r
}\right) \bigg[
\frac{\harmonic{dep}{sfh}}{\timescale{dep}}
\left(1 - e^{-t / \harmonic{dep}{sfh}}\right) -
\\
&\qquad \frac{\hharmonic{dep}{rise}{sfh}}{\timescale{dep}}
\left(e^{-t / \timescale{rise}} - e^{-t / \harmonic{dep}{sfh}}\right)
\bigg]
\end{split}
\label{eq:zalpha}
\end{equation}
\par
At~$t = 0$, the factor~$1 / (1 - e^{-t / \timescale{rise}})$ is infinite,
but the factor enclosed in square brackets is zero.
Therefore, a physical solution exists if and only if the integration
constant~$C$, included in the full derivation of this expression (see equation
\ref{eq:zalpha-full}), is equal to zero.
Because this results in the~$\infty \times 0$ indeterminate form, I can simply
define the abundance to be zero such that the boundary condition
of~$Z_\alpha(t = 0) = 0$ is satisfied.

\begin{figure*}
\centering
\includegraphics[scale = 0.52]{vartimescales.pdf}
\includegraphics[scale = 0.52]{varyieldeta.pdf}
\caption{
Analytically computed evolution in the oxygen abundance according to equation
\ref{eq:zalpha}.
\textbf{Left}: For~$y_\alpha = 0.015$ and~$\eta = 2.5$, each curve denotes a
different choice of some timescale.
With black visualizing a fiducial choice of parameters of (\timescale{rise},
\timescale{sfh},~$\tau_\star$) = (2 Gyr, 6 Gyr, 5 Gyr), crimson shows a short
rise timescale (i.e.,~$\timescale{rise} = 1$ Gyr), lime green shows a more
extended SFH (i.e.,~$\timescale{sfh} = 10$ Gyr), and blue shows a higher SFE
(i.e.,~$\tau_\star = 2$ Gyr).
\textbf{Right}: For the fiducial choice of timescales in the right-hand panel,
each curve denotes a different choice of~$y_\alpha$ and~$\eta$ as denoted in
the legend.
For each choice of~$y_\alpha$, the value of~$\eta$ is chosen such that the
ratio~$y_\alpha / (1 + \eta - r)$ is approximately constant, and vice versa in
the case of the blue line where I choose~$\eta = 0$ and compute the
corresponding value of~$y_\alpha$.
}
\label{fig:analytic-evolution}
\end{figure*}

The left panel of Fig.~\ref{fig:analytic-evolution} visualizes the evolution of
the alpha element abundances according to equation~\ref{eq:zalpha} under
different choices of timescales.
% A shorter rise timescale has the effect of raising the abundances overall
\par
The right panel of Fig.~\ref{fig:analytic-evolution} visualizes the enrichment
history for the fiducial choice of timescales in the left panel, but with
different values of~$y_\alpha$ and~$\eta$, selected such that the value of
$y_\alpha / (1 + \eta - r)$ is approximately constant.
\par\null\par\noindent
\textbf{The Time-Derivative of~$Z_\alpha(t)$}
\par\noindent
The differentiate~$Z_\alpha(t)$ with time, it compactifies notation to define
some function~$g(t)$ denoting the term in square brackets in equation
\ref{eq:zalpha}:
\begin{equation}
g(t) \equiv \frac{\harmonic{dep}{sfh}}{\timescale{dep}}
\left(1 - e^{-t / \harmonic{dep}{sfh}}\right) -
\frac{\hharmonic{dep}{rise}{sfh}}{\timescale{dep}}
\left(e^{-t / \timescale{rise}} - e^{-t / \harmonic{dep}{sfh}}\right),
\label{eq:g}
\end{equation}
such that the expression for~$Z_\alpha$ reduces to
\begin{equation}
Z_\alpha(t) = \frac{1}{1 - e^{-t / \timescale{rise}}}
\left(\frac{y_\alpha}{1 + \eta - r}\right) g(t),
\end{equation}
and its time-derivative can then be obtained with product-rule:
\begin{subequations}\begin{align}
\dot{Z}_\alpha(t) &= \left(\frac{y_\alpha}{1 + \eta - r}\right)\left[
\frac{
	-e^{-t / \timescale{rise}}
}{
	\timescale{rise} \left(1 - e^{-t / \timescale{rise}}\right)^2
} g(t) + \frac{1}{1 - e^{-t / \timescale{rise}}} \dot{g}(t)
\right]
\\
&= \frac{1}{1 - e^{-t / \timescale{rise}}}
\left(\frac{y_\alpha}{1 + \eta - r}\right) g(t)
\left[
\frac{
	-e^{-t / \timescale{rise}}
}{
	\timescale{rise} \left(1 - e^{-t / \timescale{rise}}\right)
} + \frac{\dot{g}(t)}{g(t)}\right]
\label{eq:zdotalpha}
\\
\implies \frac{\dot{Z}_\alpha(t)}{Z_\alpha(t)} &=
\frac{\dot{g}(t)}{g(t)} - \frac{
	e^{-t / \timescale{rise}}
}{
	\timescale{rise} \left(1 - e^{-t / \timescale{rise}}\right)
}.
\label{eq:zdotalpha_over_zalpha}
\end{align}\end{subequations}
The time-derivative of~$g(t)$ is straight-forward:
\begin{subequations}\begin{align}
\begin{split}
\dot{g}(t) &= \frac{\harmonic{dep}{sfh}}{\timescale{dep}}
\left(0 - e^{-t / \harmonic{dep}{sfh}} \frac{-1}{\harmonic{dep}{sfh}}\right) -
\\
&\qquad \frac{\hharmonic{dep}{rise}{sfh}}{\timescale{dep}}
\left(e^{-t / \timescale{rise}}\frac{-1}{\timescale{rise}} -
e^{-t / \harmonic{dep}{sfh}} \frac{-1}{\harmonic{dep}{sfh}}\right)
\end{split}
\\
&= \frac{e^{-t / \harmonic{dep}{sfh}}}{\timescale{dep}} +
\frac{\hharmonic{dep}{rise}{sfh}}{\timescale{dep}}
\left(\frac{e^{-t / \timescale{rise}}}{\timescale{rise}} -
\frac{e^{-t / \harmonic{dep}{sfh}}}{\harmonic{dep}{sfh}}\right)
\label{eq:gdot}
\\
\implies \frac{\dot{g}(t)}{g(t)} &= \ddfrac{
	e^{-t / \harmonic{dep}{sfh}} + \hharmonic{dep}{rise}{sfh}
	\left(\frac{e^{-t / \timescale{rise}}}{\timescale{rise}} -
	\frac{e^{-t / \harmonic{dep}{sfh}}}{\harmonic{dep}{sfh}}\right)
}{
	\harmonic{dep}{sfh}\left(1 - e^{-t / \harmonic{dep}{sfh}}\right) -
	\hharmonic{dep}{rise}{sfh}\left(e^{-t / \timescale{rise}} -
	e^{-t / \harmonic{dep}{sfh}}\right)
}.
\label{eq:gdot_over_g}
\end{align}\end{subequations}
Between equations~\ref{eq:zdotalpha},~\ref{eq:zdotalpha_over_zalpha},
\ref{eq:gdot}, and~\ref{eq:gdot_over_g}, the full solution
to~$\dot{Z}_\alpha(t)$ and~$\dot{Z}_\alpha(t) / Z_\alpha(t)$ is specified.

\par\null\par\noindent
\textbf{The Second Time-Derivative of~$Z_\alpha(t)$}
\par\noindent
Differentating~$\dot{Z}_\alpha(t)$ with time follows from
equation~\ref{eq:zdotalpha}, though it simlifies things somewhat to instead
start from equation~\ref{eq:zdotalpha_over_zalpha}:
\begin{subequations}\begin{align}
\ddot{Z}_\alpha(t) &= \frac{d}{dt} \left[Z_\alpha(t) \left(
\frac{\dot{g}(t)}{g(t)} - \frac{
	e^{-t / \timescale{rise}}
}{
	\timescale{rise} \left(1 - e^{-t / \timescale{rise}}\right)
}\right)\right]
\\
&= \dot{Z}_\alpha(t) \left(\frac{\dot{g}(t)}{g(t)} - \frac{
	e^{-t / \timescale{rise}}
}{
	\timescale{rise} \left(1 - e^{-t / \timescale{rise}}\right)
}\right) + Z_\alpha(t) \frac{d}{dt} \left(\frac{\dot{g}(t)}{g(t)} - \frac{
	e^{-t / \timescale{rise}}
}{
	\timescale{rise} \left(1 - e^{-t / \timescale{rise}}\right)
}\right)
\\
\begin{split}
&= Z_\alpha(t) \Bigg[
\left(\frac{\dot{g}(t)}{g(t)} - \frac{
	e^{-t / \timescale{rise}}
}{
	\timescale{rise} \left(1 - e^{-t / \timescale{rise}}\right)
}\right)^2 + \frac{
	g(t) \ddot{g}(t) - \dot{g}(t)^2
}{
	g(t)^2
} -
\\
&\qquad \frac{
	e^{-t / \timescale{rise}}
}{
	\timescale{rise}^2 \left(1 - e^{-t / \timescale{rise}}\right)^2
}
\Bigg]
\end{split}
\end{align}
\\
\begin{align}
\begin{split}
&= Z_\alpha(t) \Bigg[
\left(\frac{\dot{g}(t)}{g(t)}\right)^2 - \frac{
	2 e^{-t / \timescale{rise}}
}{
	\timescale{rise} \left(1 - e^{-t / \timescale{rise}}\right)
}\left(\frac{\dot{g}(t)}{g(t)}\right) + \frac{
	e^{-2t / \timescale{rise}}
}{
	\timescale{rise}^2 \left(1 - e^{-t / \timescale{rise}}\right)^2
} +
\\
&\qquad \frac{\ddot{g}(t)}{g(t)} -
\left(\frac{\dot{g}(t)}{g(t)}\right)^2 + \frac{
	e^{-t / \timescale{rise}}
}{
	\timescale{rise}^2 \left(1 - e^{-t / \timescale{rise}}\right)^2
}
\Bigg]
\end{split}
\\
\implies \frac{\ddot{Z}_\alpha(t)}{Z_\alpha(t)} &= \left[\frac{
	e^{-t / \timescale{rise}} + e^{-2t / \timescale{rise}}
}{
	\timescale{rise}^2 \left(1 - e^{-t / \timescale{rise}}\right)^2
} - \frac{
	2 e^{-t / \timescale{rise}}
}{
	\timescale{rise} \left(1 - e^{-t / \timescale{rise}}\right)
} \left(\frac{\dot{g}(t)}{g(t)}\right) +
\frac{\ddot{g}(t)}{g(t)}
\right],
\label{eq:zdotdotalpha_over_zalpha}
\end{align}\end{subequations}
and~$\ddot{g}(t)$ follows from differentiating equation~\ref{eq:gdot}:
\begin{equation}
\ddot{g}(t) = \frac{
	-e^{-t / \harmonic{dep}{sfh}}
}{
	\timescale{dep}\harmonic{dep}{sfh}
} - \frac{
	\hharmonic{dep}{rise}{sfh}
}{
	\timescale{dep}
} \left(\frac{
	e^{-t / \timescale{rise}}
}{
	\timescale{rise}^2
} - \frac{
	e^{-t / \harmonic{dep}{sfh}}
}{
	\harmonic{dep}{sfh}^2
}\right).
\label{eq:gdotdot}
\end{equation}
Combining equation~\ref{eq:zdotdotalpha_over_zalpha} with equation
\ref{eq:zdotalpha_over_zalpha} yields the expression for~$\ddot{Z}_\alpha(t) /
\dot{Z}_\alpha(t)$ needed for the MDF optimization criterion (equation
\ref{eq:mhmdf_maxima_criterion}):
\begin{subequations}\begin{align}
\frac{\ddot{Z}_\alpha(t)}{\dot{Z}_\alpha(t)} &=
\frac{\ddot{Z}_\alpha(t)}{Z_\alpha(t)}
\left(\frac{\dot{Z}_\alpha(t)}{Z_\alpha(t)}\right)^{-1}
\\
&= \ddfrac{
	\frac{
		e^{-t / \timescale{rise}} + e^{-2t / \timescale{rise}}
	}{
		\timescale{rise}^2 \left(1 - e^{-t / \timescale{rise}}\right)^2
	} - \frac{
		2 e^{-t / \timescale{rise}}
	}{
		\timescale{rise} \left(1 - e^{-t / \timescale{rise}}\right)
	} \left(\frac{\dot{g}(t)}{g(t)}\right) +
	\frac{\ddot{g}(t)}{g(t)}
}{
	\frac{\dot{g}(t)}{g(t)} - \frac{
		e^{-t / \timescale{rise}}
	}{
		\timescale{rise} \left(1 - e^{-t / \timescale{rise}}\right)
	}
}.
\label{eq:zdotdotalpha_over_zdotalpha}
\end{align}\end{subequations}
This expression does not simplify any further, or at least not in a way that is
useful.

















\newpage
\bibliographystyle{mnras}
\bibliography{onezone-mdfs}

\newpage
\noindent
\textbf{Full Solution to Equation~\ref{eq:za-linear-ode}}

\begin{subequations}\begin{align}
\begin{split} % a
Z_\alpha(t) &= \exp\left(
\frac{-t}{\timescale{dep}} - \ln (1 - e^{-t / \timescale{rise}}) +
\frac{t}{\timescale{sfh}}
\right)
\\
&\qquad \left[
\int_0^t \exp\left(
\frac{t'}{\timescale{dep}} + \ln (1 - e^{-t / \timescale{rise}}) -
\frac{t'}{\timescale{sfh}}
\right)
\frac{y_\alpha}{\tau_\star} dt' + C
\right]
\end{split}
\\
\begin{split} % b
&= \frac{1}{1 - e^{-t / \timescale{rise}}}
\exp \left(-t\frac{
	\timescale{sfh} - \timescale{dep}
}{
	\timescale{sfh}\timescale{dep}
}\right) \frac{y_\alpha}{\tau_\star}
\\
&\qquad \left[ \int_0^t (1 - e^{-t' / \timescale{rise}}) \exp \left(
t' \frac{\timescale{sfh} - \timescale{dep}}{\timescale{sfh} \timescale{dep}}
\right)dt' + C\right]
\end{split}
\\
\begin{split} % c
&= \frac{1}{1 - e^{-t / \timescale{rise}}}
\exp \left(-t\frac{
	\timescale{sfh} - \timescale{dep}
}{
	\timescale{sfh}\timescale{dep}
}\right) \frac{y_\alpha}{\tau_\star}
\\
&\qquad \left[ \int_0^t \left(
\exp\left(t' \frac{
	\timescale{sfh} - \timescale{dep}
}{
	\timescale{sfh}\timescale{dep}
}\right)
- \exp\left(
t' \frac{
	\timescale{sfh} - \timescale{dep}
}{
	\timescale{sfh}\timescale{dep}
} - \frac{t}{\timescale{rise}}
\right)
\right) dt' + C\right]
\end{split}
\\
\begin{split} % d
&= \frac{1}{1 - e^{-t / \timescale{rise}}}
\exp \left(-t\frac{
	\timescale{sfh} - \timescale{dep}
}{
	\timescale{sfh}\timescale{dep}
}\right) \frac{y_\alpha}{\tau_\star} \bigg[ \int_0^t
\exp\left(t'\frac{
	\timescale{sfh} - \timescale{dep}
}{
	\timescale{sfh}\timescale{dep}
} \right) dt' -
\\
&\qquad
\int_0^t \exp\left(t'\frac{
	\timescale{sfh}\timescale{rise} - \timescale{dep}\timescale{rise} -
	\timescale{sfh}\timescale{dep}
}{
	\timescale{sfh}\timescale{dep}\timescale{rise}
}\right) dt' + C \bigg]
\end{split}
\\
\begin{split} % e
&= \frac{1}{1 - e^{-t / \timescale{rise}}}
\exp \left(-t\frac{
	\timescale{sfh} - \timescale{dep}
}{
	\timescale{sfh}\timescale{dep}
}\right) \frac{y_\alpha}{\tau_\star} \bigg[\frac{
	\timescale{sfh}\timescale{dep}
}{
	\timescale{sfh} - \timescale{dep}
} \exp \left(t' \frac{
	\timescale{sfh} - \timescale{dep}
}{
	\timescale{sfh}\timescale{dep}
}\right) \bigg|_0^t -
\\
&\qquad \frac{
	\timescale{sfh}\timescale{dep}\timescale{rise}
}{
	\timescale{sfh}\timescale{rise} - \timescale{dep}\timescale{rise} -
	\timescale{sfh}\timescale{dep}
} \exp\left(
t'\frac{
	\timescale{sfh}\timescale{rise} - \timescale{dep}\timescale{rise} -
	\timescale{sfh}\timescale{dep}
}{
	\timescale{sfh}\timescale{dep}\timescale{rise}
}
\right)\bigg|_0^t + C \bigg]
\end{split}
\\
\begin{split} % f
&= \frac{1}{1 - e^{-t / \timescale{rise}}}
\exp \left(-t\frac{
	\timescale{sfh} - \timescale{dep}
}{
	\timescale{sfh}\timescale{dep}
}\right) \frac{y_\alpha}{\tau_\star} \bigg[\frac{
	\timescale{sfh}\timescale{dep}
}{
	\timescale{sfh} - \timescale{dep}
} \left(
\exp\left(
t\frac{
	\timescale{sfh} - \timescale{dep}
}{
	\timescale{sfh}\timescale{dep}
}\right) - 1\right) -
\\
&\qquad \frac{
	\timescale{sfh}\timescale{dep}\timescale{rise}
}{
	\timescale{sfh}\timescale{rise} - \timescale{dep}\timescale{rise} -
	\timescale{sfh}\timescale{dep}
} \left(
\exp\left(t\frac{
	\timescale{sfh}\timescale{rise} - \timescale{dep}\timescale{rise} -
	\timescale{sfh}\timescale{dep}
}{
	\timescale{sfh}\timescale{dep}\timescale{rise}
}\right) - 1\right)
\\
&\qquad + C\bigg]
\end{split}
\\
\begin{split} % g
&= \frac{1}{1 - e^{-t / \timescale{rise}}}
\left(\frac{y_\alpha}{\tau_\star}\right)
\bigg[\frac{
	\timescale{sfh}\timescale{dep}
}{
	\timescale{sfh} - \timescale{dep}
} \left(
1 - \exp\left(-t\frac{
	\timescale{sfh} - \timescale{dep}
}{
	\timescale{sfh}\timescale{dep}
}\right)
\right) -
\\
&\qquad \frac{
	\timescale{sfh}\timescale{dep}\timescale{rise}
}{
	\timescale{sfh}\timescale{rise} - \timescale{dep}\timescale{rise} -
	\timescale{sfh}\timescale{dep}
} \left(e^{-t / \timescale{rise}} -
\exp\left(-t
\frac{
	\timescale{sfh} - \timescale{dep}
}{
	\timescale{sfh}\timescale{dep}
}
\right)
\right) + C\bigg]
\end{split}
\\
\begin{split} % h
&= \frac{1}{1 - e^{-t / \timescale{rise}}}
\left(\frac{y_\alpha}{1 + \eta - r}\right)
\bigg[\frac{
	\timescale{sfh}
}{
	\timescale{sfh} - \timescale{dep}
} \left(
1 - \exp\left(-t\frac{
	\timescale{sfh} - \timescale{dep}
}{
	\timescale{sfh}\timescale{dep}
}\right)
\right) -
\\
&\qquad \frac{
	\timescale{sfh}\timescale{rise}
}{
	\timescale{sfh}\timescale{rise} - \timescale{dep}\timescale{rise} -
	\timescale{sfh}\timescale{dep}
} \left(e^{-t / \timescale{rise}} -
\exp\left(-t
\frac{
	\timescale{sfh} - \timescale{dep}
}{
	\timescale{sfh}\timescale{dep}
}
\right)
\right) + C\bigg].
\end{split}
\label{eq:zalpha-full}
\end{align}\end{subequations}




















\newpage
\noindent
\textbf{Useful Identities}
\begin{subequations}\begin{align}
\frac{d}{dt}\left(1 - e^{-t / \timescale{rise}}\right)^{-1} &=
-\left(1 - e^{-t / \timescale{rise}}\right)^{-2}
\frac{d}{dt}\left(1 - e^{-t / \timescale{rise}}\right)
\\
&= \frac{-1}{\left(1 - e^{-t / \timescale{rise}}\right)^2}
\left(0 - e^{-t / \timescale{rise}}\left(\frac{-1}{\timescale{rise}}\right)
\right)
\\
&= \frac{
	-e^{-t / \timescale{rise}}
}{
	\timescale{rise} \left(1 - e^{-t / \timescale{rise}}\right)^2
}
\end{align}\end{subequations}

\par\null\par
\begin{center}
\rule[0.7\baselineskip]{0.5\textwidth}{0.4pt}
\end{center}

\begin{subequations}\begin{align}
\frac{d}{dt} \left(\frac{
	e^{-t / \timescale{rise}}
}{
	1 - e^{-t / \timescale{rise}}
}\right) &= \frac{
	\left(1 - e^{-t / \timescale{rise}}\right) e^{-t / \timescale{rise}}
	\left(-1 / \timescale{rise}\right) -
	e^{-t / \timescale{rise}} \left(-e^{-t / \timescale{rise}}
	\left(-1 / \timescale{rise}\right)\right)
}{
	\left(1 - e^{-t / \timescale{rise}}\right)^2
}
\\
&= \frac{
	-e^{-t / \timescale{rise}} + e^{-2t / \timescale{rise}} -
	e^{-2t / \timescale{rise}}
}{
	\timescale{rise} \left(1 - e^{-t / \timescale{rise}}\right)^2
}
\\
&= \frac{
	-e^{-t / \timescale{rise}}
}{
	\timescale{rise} \left(1 - e^{-t / \timescale{rise}}\right)^2
}
\end{align}\end{subequations}

\par\null\par
\begin{center}
\rule[0.7\baselineskip]{0.5\textwidth}{0.4pt}
\end{center}

\begin{subequations}\begin{align}
\begin{split}
\frac{d}{dt}\left(\frac{
	-e^{-t / \timescale{rise}}
}{
	\left(1 - e^{-t / \timescale{rise}}\right)^2
}\right) &= \frac{1}{(1 - e^{-t / \timescale{rise}})^4}\bigg[
\left(1 - e^{-t / \timescale{rise}}\right)^2
\left(-e^{-t / \timescale{rise}}\right)
\left(-1 / \timescale{rise}\right) +
\\
&\qquad e^{-t / \timescale{rise}}
2\left(1 - e^{-t / \timescale{rise}}\right)
\left(0 - e^{-t / \timescale{rise}}\left(-1 / \timescale{rise}\right)\right)
\bigg]
\end{split}
\\
&= \frac{
	e^{-t / \timescale{rise}}
	\left(1 - e^{-t / \timescale{rise}}\right)^2 +
	2e^{-2t / \timescale{rise}}
	\left(1 - e^{-t / \timescale{rise}}\right)
}{
	\timescale{rise} \left(1 - e^{-t / \timescale{rise}}\right)^4
}
\\
&= \frac{
	e^{-t / \timescale{rise}} \left(1 - e^{-t / \timescale{rise}}\right) +
	2e^{-2t / \timescale{rise}}
}{
	\timescale{rise}\left(1 - e^{-t / \timescale{rise}}\right)^3
}
\\
&= \frac{
	e^{-t / \timescale{rise}} - e^{-2t / \timescale{rise}} +
	2e^{-2t / \timescale{rise}}
}{
	\timescale{rise} \left(1 - e^{-t / \timescale{rise}}\right)^3
}
\\
&= \frac{
	e^{-t / \timescale{rise}} + e^{-2t / \timescale{rise}}
}{
	\timescale{rise} \left(1 - e^{-t / \timescale{rise}}\right)^3
}
\end{align}\end{subequations}

% \par\null\par
% \begin{center}
% \rule[0.7\baselineskip]{0.5\textwidth}{0.4pt}
% \end{center}
\newpage

\begin{subequations}\begin{align}
\int_0^t t' e^{\alpha t'} dt' &= \int_0^t \frac{\partial}{\partial \alpha}
e^{\alpha t'} dt'
\\
&= \frac{\partial}{\partial \alpha} \int_0^t e^{\alpha t'} dt'
\\
&= \frac{\partial}{\partial \alpha} \left[
\frac{1}{\alpha} e^{\alpha t'} \Bigg|_0^t\right]
\\
&= \frac{\partial}{\partial \alpha} \left[
\frac{1}{\alpha} \left(e^{\alpha t} - 1\right)\right]
\\
&= \frac{-1}{\alpha^2} \left(e^{\alpha t} - 1\right) +
\frac{1}{\alpha} te^{\alpha t}
\\
&= \frac{1}{\alpha^2} + e^{\alpha t} \left(\frac{t}{\alpha} -
\frac{1}{\alpha^2}\right)
\end{align}\end{subequations}

\end{document}




% \textbf{Time-Derivative of~$Z_\alpha(t)$}
% \par\noindent
% Taking the time-derivative of equation~\ref{eq:zalpha} first and foremost
% requires product rule.
% It is helpful to split this process up into pieces:
% \begin{subequations}\begin{align}
% \frac{d}{dt} (1 - e^{-t / \timescale{rise}})^{-1} &=
% -(1 - e^{-t / \timescale{rise}})^{-2}\frac{d}{dt}e^{-t / \timescale{rise}}
% \\
% &= \frac{
% 	e^{-t / \timescale{rise}}
% }{
% 	\timescale{rise} (1 - e^{-t / \timescale{rise}})^2
% }.
% \end{align}\end{subequations}
% To save space, I define the function~$g(t)$ to denote the term in square
% brackets in equation~\ref{eq:zalpha}.
% Its time derivative:
% \begin{subequations}\begin{align}
% \begin{split} % a
% \dot{g}(t) &= \frac{\timescale{sfh}}{\timescale{sfh} - \timescale{dep}}
% \exp \left( -t \frac{
% 	\timescale{sfh} - \timescale{dep}
% }{
% 	\timescale{sfh}\timescale{dep}
% }\right)\frac{
% 	\timescale{sfh} - \timescale{dep}
% }{
% 	\timescale{sfh}\timescale{dep}
% } - 
% \\
% &\qquad \frac{
% 	\timescale{sfh}\timescale{rise}
% }{
% 	\timescale{sfh}\timescale{rise} - \timescale{dep}\timescale{rise} -
% 	\timescale{sfh}\timescale{dep}
% } \bigg(
% \frac{-1}{\timescale{rise}} e^{-t / \timescale{rise}} +
% \\
% &\qquad
% \exp \left( -t \frac{
% 	\timescale{sfh} - \timescale{dep}
% }{
% 	\timescale{sfh}\timescale{dep}
% }
% \right)
% \frac{
% 	\timescale{sfh} - \timescale{dep}
% }{
% 	\timescale{sfh}\timescale{dep}
% }
% \bigg)
% \end{split}
% \\
% \begin{split} % b
% &= \frac{1}{\timescale{dep}}\exp\left( -t \frac{
% 	\timescale{sfh} - \timescale{dep}
% }{
% 	\timescale{sfh}\timescale{dep}
% }\right) + \frac{
% 	\timescale{sfh}
% }{
% 	\timescale{sfh}\timescale{rise} - \timescale{dep}\timescale{rise} -
% 	\timescale{sfh}\timescale{dep}
% } e^{-t / \timescale{rise}} -
% \\
% &\qquad \frac{
% 	\timescale{rise}\timescale{sfh} - \timescale{rise}\timescale{dep}
% }{
% 	\timescale{sfh}\timescale{rise}\timescale{dep} - \timescale{dep}^2
% 	\timescale{rise} - \timescale{sfh}\timescale{dep}^2
% } \exp \left( -t \frac{
% 	\timescale{sfh} - \timescale{dep}
% }{
% 	\timescale{sfh}\timescale{dep}
% }
% \right)
% \end{split}
% \\
% \begin{split} % c
% &= \exp\left(-t \frac{
% 	\timescale{sfh} - \timescale{dep}
% }{
% 	\timescale{sfh}\timescale{dep}
% }\right) \bigg[
% \frac{1}{\timescale{dep}} - \frac{
% 	\timescale{rise}\timescale{sfh} - \timescale{rise}\timescale{dep}
% }{
% 	\timescale{sfh}\timescale{rise}\timescale{dep} - \timescale{dep}^2
% 	\timescale{rise} - \timescale{sfh}\timescale{dep}^2
% } +
% \\
% &\qquad \frac{
% 	\timescale{sfh}
% }{
% 	\timescale{sfh}\timescale{rise} - \timescale{dep}\timescale{rise} -
% 	\timescale{sfh}\timescale{dep}
% } \exp \left(t \frac{
% 	\timescale{rise}\timescale{sfh} - \timescale{dep}\timescale{rise} -
% 	\timescale{sfh}\timescale{dep}
% }{
% 	\timescale{rise}\timescale{sfh}\timescale{dep}
% }\right) \bigg]
% \end{split}
% \\
% \begin{split} % d
% &= \exp\left(-t \frac{
% 	\timescale{sfh} - \timescale{dep}
% }{
% 	\timescale{sfh}\timescale{dep}
% }\right) \frac{
% 	\timescale{sfh}
% }{
% 	\timescale{sfh}\timescale{rise} - \timescale{dep}\timescale{rise} -
% 	\timescale{sfh}\timescale{dep}
% }
% \\
% &\qquad \left[\exp\left(t \frac{
% 	\timescale{rise}\timescale{sfh} - \timescale{dep}\timescale{rise} -
% 	\timescale{sfh}\timescale{dep}
% }{
% 	\timescale{rise}\timescale{sfh}\timescale{dep}
% }\right)
% - 1\right].
% \end{split}
% \end{align}\end{subequations}
% At this point, it simplifies notation to define
% \begin{equation}
% \hharmonic{dep}{sfh}{rise} \equiv \frac{
% 	\timescale{rise}\timescale{sfh}\timescale{dep}
% }{
% 	\timescale{sfh}\timescale{rise} - \timescale{dep}\timescale{rise} -
% 	\timescale{sfh}\timescale{dep}
% } = \left( \frac{1}{\timescale{dep}} - \frac{1}{\timescale{sfh}} -
% \frac{1}{\timescale{rise}}\right)^{-1},
% \end{equation}
% which is reminiscent of the harmonic timescales~$\harmonic{X}{Y}$ seen
% in~\citet*{Weinberg2017}, but for three timescales instead of two.
% Adopting this notation yields the following expression for~$\dot{g}(t)$:
% \begin{equation}
% \begin{split}
% \dot{g}(t) &= \frac{
% 	\hharmonic{dep}{rise}{sfh}
% }{
% 	\timescale{rise}\timescale{dep}
% } e^{-t / \harmonic{dep}{sfh}} \left(e^{t /
% \hharmonic{dep}{rise}{sfh}} - 1\right)
% \\
% &= \frac{
% 	\hharmonic{dep}{rise}{sfh}
% }{
% 	\timescale{rise}\timescale{dep}
% } \left(e^{-t / \timescale{rise}} - e^{-t / \harmonic{dep}{sfh}}\right).
% \end{split}
% \end{equation}
% I can now write the full expression for~$\dot{Z}_\alpha(t)$:
% \begin{subequations}\begin{align}
% \dot{Z}_\alpha(t) &= \left(\frac{y_\alpha}{1 + \eta - r}\right)
% \left[
% \frac{e^{-t / \timescale{rise}}}{
% 	\timescale{rise} (1 - e^{-t / \timescale{rise}})^2
% } g(t) + \frac{1}{1 - e^{-t / \timescale{rise}}} \dot{g}(t)
% \right]
% \\
% &= \left(\frac{y_\alpha}{1 + \eta - r}\right)
% \frac{1}{1 - e^{-t / \timescale{rise}}} \left[
% \dot{g}(t) + \frac{
% 	e^{-t / \timescale{rise}}
% }{
% 	\timescale{rise} (1 - e^{-t / \timescale{rise}})
% } g(t)
% \right],
% \end{align}\end{subequations}
% and an expression for~$\dot{Z}_\alpha(t) / Z_\alpha(t)$:
% \begin{subequations}\begin{align}
% \frac{\dot{Z}_\alpha(t)}{Z_\alpha(t)} &= \ddfrac{
% 	\dot{g}(t) + \frac{
% 		e^{-t / \timescale{rise}}
% 	}{
% 		\timescale{rise} (1 - e^{-t / \timescale{rise}})
% 	} g(t)
% }{
% 	g(t)
% }
% \\
% &= \frac{\dot{g}(t)}{g(t)} + \frac{
% 	e^{-t / \timescale{rise}}
% }{
% 	\timescale{rise} (1 - e^{-t / \timescale{rise}})
% }.
% \end{align}\end{subequations}
% Expanding on $\dot{g}(t) / g(t)$:
% \begin{subequations}\begin{align}
% \frac{\dot{g}(t)}{g(t)} &= \left(\frac{g(t)}{\dot{g}(t)}\right)^{-1}
% \\
% &= \left[\ddfrac{
% 	\frac{
% 		\harmonic{dep}{sfh}
% 	}{
% 		\timescale{dep}
% 	} \left(1 - e^{-t / \harmonic{dep}{sfh}}\right) -
% 	\frac{
% 		\hharmonic{dep}{rise}{sfh}
% 	}{
% 		\timescale{dep}
% 	} \left(e^{-t / \timescale{rise}} -
% 	e^{-t / \harmonic{dep}{sfh}}\right)
% }{
% 	\frac{
% 		\hharmonic{dep}{rise}{sfh}
% 	}{
% 		\timescale{rise}\timescale{dep}
% 	}
% 	\left(e^{-t / \timescale{rise}} - e^{-t / \harmonic{dep}{sfh}}
% 	\right)
% }\right]^{-1}
% \\
% &= \frac{1}{\timescale{rise}}\left[
% \frac{
% 	\harmonic{dep}{sfh}
% }{
% 	\hharmonic{dep}{rise}{sfh}
% }\left(
% \frac{
% 	1 - e^{-t / \harmonic{dep}{sfh}}
% }{
% 	e^{-t / \timescale{rise}} - e^{-t / \harmonic{dep}{sfh}}
% }
% \right) - 1\right]^{-1},
% \end{align}\end{subequations}
% and therefore the final expression for~$\dot{Z}_\alpha(t) / Z_\alpha(t)$:
% \begin{equation}
% \frac{\dot{Z}_\alpha(t)}{Z_\alpha(t)} = \frac{
% 	e^{-t / \timescale{rise}}
% }{
% 	\timescale{rise} (1 - e^{-t / \timescale{rise}})
% } + \frac{1}{\timescale{rise}} \left[
% \frac{
% 	\harmonic{dep}{sfh}
% }{
% 	\hharmonic{dep}{rise}{sfh}
% }\left(\frac{
% 	1 - e^{-t / \harmonic{dep}{sfh}}
% }{
% 	e^{-t / \timescale{rise}} - e^{-t / \harmonic{dep}{sfh}}
% }\right) - 1\right]^{-1}
% \end{equation}

