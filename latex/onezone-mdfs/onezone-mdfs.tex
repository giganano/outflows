
\documentclass[12pt]{article}
\usepackage[margin = 1in]{geometry}
\usepackage[dvipsnames]{xcolor}
\usepackage{tabularx}
\usepackage{graphicx}
\usepackage{enumitem}
\usepackage{hyperref}
% \usepackage{fancyhdr}
\usepackage[margin = 0cm]{caption}
\usepackage{wrapfig}
\usepackage{amsmath}
\usepackage{amssymb}
\usepackage{natbib}
\setlength{\bibsep}{0pt}
\hypersetup{ % all blue links
	colorlinks		= true,
	urlcolor 		= blue,
	linkcolor 		= blue,
	citecolor 		= blue
}
\setcitestyle{aysep={}, yysep={,}}
\newcommand{\aj}{AJ}
\newcommand{\aap}{A\&A}
\newcommand{\aapr}{A\&ARv}
\newcommand{\apj}{ApJ}
\newcommand{\apjl}{ApJL}
\newcommand{\apjs}{ApJS}
\newcommand{\apss}{Ap\&SS}
\newcommand{\araa}{ARA\&A}
\newcommand{\baas}{BAAS}
\newcommand{\fcp}{Fundamentals Cosmic Phys.}
\newcommand{\mnras}{MNRAS}
\newcommand{\nat}{Nature}
\newcommand{\pasa}{PASA}
\newcommand{\pasp}{PASP}
\newcommand{\ddfrac}[2]{\frac{\displaystyle{#1}}{\displaystyle{#2}}}
\newcommand{\msun}{\ensuremath{\text{M}_\odot}}
\newcommand{\scinote}[2]{\ensuremath{#1\times10^{#2}}}
\providecommand{\noopsort}[1]{}

\newcommand{\timescale}[1]{\ensuremath{\tau_\text{#1}}}

\begin{document}

\begin{center}
\textbf{The Shape and Centroid of Abundance Distributions for Inside-Out Star
Formation Histories in One-Zone Models}
\par\null\par
James W. Johnson
\par\null\par
\rule[0.7\baselineskip]{0.5\textwidth}{0.4pt}
\end{center}

\par\noindent
\textbf{Analytic solution to~$Z_\alpha(t)$}
\par\noindent
The goal in this section is to obtain an analytic expression for the evolution
of the alpha element\footnote{
	As usual, an alpha element refers to, e.g., O or Mg. To be exact, it refers
	to an element whose only statistically significant enrichment source is a
	metallicity-independent population-averaged yield from massive stars.
} abundances~$Z_\alpha(t) \equiv M_\alpha / M_\text{g}$ using arguments similar
to those of~\citet*{Weinberg2017}.
For the purposes of these analytic approximations, I assume that the star
formation efficiency (SFE) timescale
$\tau_\star \equiv M_\text{g} / \dot{M}_\star$ and the outflow mass loading
factor~$\eta \equiv \dot{M}_\text{out} / \dot{M}_\star$ are both constant in
time.
Ultimately, this will be used to compute the metallicity distribution function
(MDF) according to
\begin{equation}
\frac{dN}{dZ_\alpha} = \frac{dN}{dt}/\frac{dZ_\alpha}{dt}
\propto \frac{\dot{M}_\star(t)}{\dot{Z}_\alpha(t)}.
\label{dndz}
\end{equation}
\par
For alpha elements,~\citet*{Weinberg2017} express the rate of change of the
mass present in the ISM as
\begin{equation}
\dot{M}_\alpha = y_\alpha \dot{M}_\star - Z_\alpha \dot{M}_\star (1 + \eta - r),
\label{eq:mdot_alpha}
\end{equation}
where~$r$ is a term which accounts for the return of stellar envelopes back to
the ISM.
Here I derive~$\dot{Z}_\alpha(t)$ assuming the ``inside-out'' star formation
history (SFH) from~\citet{Johnson2021}:
\begin{equation}
\dot{M}_\star(t) \propto (1 - e^{-t / \timescale{rise}})
e^{-t / \timescale{sfh}}.
\label{eq:insideout_sfh}
\end{equation}
The physically interesting advantage of this parameterization of the SFH over
the classic ``linear-times-exponential''~$t e^{-t / \timescale{sfh}}$ form is
that~$\timescale{rise}$ provides one with some control over how long the SFH is
increasing independent of~$\timescale{sfh}$.
\par
Based on the definition of~$Z_\alpha$, the quotient rule implies that its
time-derivative can be expressed as
\begin{subequations}\begin{align}
\dot{Z}_\alpha &= \frac{
	M_\text{g} \dot{M}_\alpha - M_\alpha \dot{M}_\text{g}
}{
	M_\text{g}^2
}
\\
&= \frac{
	M_\text{g} (y_\alpha \dot{M}_\star - Z_\alpha \dot{M}_\star (1 + \eta - r))
	- M_\alpha \ddot{M}_\star \tau_\star
}{
	M_\text{g}^2
}
\\
&= y_\alpha \frac{\dot{M}_\star}{M_\text{g}} -
Z_\alpha \frac{\dot{M}_\star}{M_\text{g}}(1 + \eta - r) -
Z_\alpha \frac{\ddot{M}_\star}{M_\text{g}} \tau_\star
\\
&= \frac{y_\alpha}{\tau_\star} -
Z_\alpha \left(\frac{1 + \eta - r}{\tau_\star} +
\frac{\ddot{M}_\star}{\dot{M}_\star}\right).
\label{eq:dzdt}
\end{align}\end{subequations}
The next step is to differentiate the SFH with time to compute
$\ddot{M}_\star / \dot{M}_\star$. Taking~$A$ to denote the overall normalizing
factor:
\begin{subequations}\begin{align}
\ddot{M}_\star &= \frac{d}{dt}
A(1 - e^{-t / \timescale{rise}}) e^{-t / \timescale{sfh}}
\\
&= A \frac{1}{\timescale{rise}}e^{-t / \timescale{rise}}
e^{-t / \timescale{sfh}} +
A(1 - e^{-t / \timescale{rise}}) \frac{-1}{\timescale{sfh}}
e^{-t / \timescale{sfh}}
\\
&= A(1 - e^{-t / \timescale{rise}})e^{-t / \timescale{sfh}}
\left(\frac{
	e^{-t / \timescale{rise}}
}{
	\timescale{rise}(1 - e^{-t / \timescale{rise}})
} - \frac{1}{\timescale{sfh}}
\right)
\\
\implies \frac{\ddot{M}_\star}{\dot{M}_\star} &= \frac{
	e^{-t / \timescale{rise}}
}{
	\timescale{rise}(1 - e^{-t / \timescale{rise}})
} - \frac{1}{\timescale{sfh}}.
\end{align}\end{subequations}
Plugging this into equation~\ref{eq:dzdt} above yields the following linear
ODE for~$Z_\alpha$:
\begin{equation}
\dot{Z}_\alpha + Z_\alpha \left(\frac{1}{\timescale{dep}} +
\frac{
	e^{-t / \timescale{rise}}
}{
	\timescale{rise}(1 - e^{-t / \timescale{rise}})
} - \frac{1}{\timescale{sfh}}\right) = \frac{y_\alpha}{\tau_\star},
\end{equation}
where I have substituted for the depletion time, defined as
$\timescale{dep} \equiv \tau_\star / (1 + \eta - r)$.
$\timescale{dep}$ quantifies the e-folding timescale on which the ISM would
decline due to both star formation and mass loading, if present.
For notational convenience, I define the function~$f(t)$ to denote the
multiplicative factor on~$Z_\alpha$:
\begin{equation}
f(t) \equiv \frac{1}{\timescale{dep}} + \frac{
	e^{-t / \timescale{rise}}
}{
	\timescale{rise}(1 - e^{-t / \timescale{rise}})
} - \frac{1}{\timescale{sfh}},
\end{equation}
such that the general solution for~$Z_\alpha$ can be expressed as
\begin{equation}
Z_\alpha(t) = \exp\left(-\int f(t') dt'\right)\left(
\int_0^t \exp\left(\int f(t') dt'\right) \frac{y_\alpha}{\tau_\star} dt' + C
\right),
\label{eq:za_linear_ode}
\end{equation}
where the constant~$C$ will later be assigned such that the initial
condition~$Z_\alpha(t = 0) = 0$ is satisfied.
At this point, it is helpful to zoom in on the integral of~$f(t)$.
Its term depending on~$\timescale{rise}$ can be integrated with a variable
substitution:
\begin{subequations}\begin{align}
\int \frac{
	e^{-t / \timescale{rise}}
}{
	\timescale{rise}(1 - e^{-t / \timescale{rise}})
} dt
&= \int \frac{e^{-u}}{1 - e^{-u}} du
\\
u &= \frac{t}{\timescale{rise}}
\\
du &= \frac{1}{\timescale{rise}} dt
\\
&= \ln (1 - e^{-u})
\\
&= \ln (1 - e^{-t / \timescale{rise}})
\\
\implies \int f(t) dt
&= \frac{t}{\timescale{dep}} + \ln (1 - e^{-t / \timescale{rise}}) -
\frac{t}{\timescale{sfh}}.
\end{align}\end{subequations}
Plugging this into equation~\ref{eq:za_linear_ode} above yields the following
expression for~$Z_\alpha$:
\begin{subequations}\begin{align}
\begin{split}
Z_\alpha(t) &= \exp\left(
\frac{-t}{\timescale{dep}} - \ln (1 - e^{-t / \timescale{rise}}) +
\frac{t}{\timescale{sfh}}
\right)
\\
&\qquad \left(
\int_0^t \exp\left(
\frac{t'}{\timescale{dep}} + \ln (1 - e^{-t / \timescale{rise}}) -
\frac{t'}{\timescale{sfh}}
\right)
\frac{y_\alpha}{\tau_\star} dt' + C
\right)
\end{split}
\\
\begin{split}
&= \frac{1}{1 - e^{-t / \timescale{rise}}}
\exp \left(-t\frac{
	\timescale{sfh} - \timescale{dep}
}{
	\timescale{sfh}\timescale{dep}
}\right) \frac{y_\alpha}{\tau_\star}
\\
&\qquad \left( \int_0^t (1 - e^{-t' / \timescale{rise}}) \exp \left(
t' \frac{\timescale{sfh} - \timescale{dep}}{\timescale{sfh} \timescale{dep}}
\right)dt' + C
\right)
\end{split}
\\
\begin{split}
&= \frac{1}{1 - e^{-t / \timescale{rise}}}
\exp \left(-t\frac{
	\timescale{sfh} - \timescale{dep}
}{
	\timescale{sfh}\timescale{dep}
}\right) \frac{y_\alpha}{\tau_\star} \bigg( \int_0^t
\exp\left(t'\frac{
	\timescale{sfh} - \timescale{dep}
}{
	\timescale{sfh}\timescale{dep}
} \right) dt' +
\\
&\qquad
\int_0^t \exp\left(t'\frac{
	\timescale{sfh}\timescale{rise} - \timescale{dep}\timescale{rise} -
	\timescale{sfh}\timescale{dep}
}{
	\timescale{sfh}\timescale{dep}\timescale{rise}
}\right) dt' + C \bigg)
\end{split}
\\
\begin{split}
&= \frac{1}{1 - e^{-t / \timescale{rise}}}
\exp \left(-t\frac{
	\timescale{sfh} - \timescale{dep}
}{
	\timescale{sfh}\timescale{dep}
}\right) \frac{y_\alpha}{\tau_\star} \bigg(\frac{
	\timescale{sfh}\timescale{dep}
}{
	\timescale{sfh} - \timescale{dep}
} \exp \left(t' \frac{
	\timescale{sfh} - \timescale{dep}
}{
	\timescale{sfh}\timescale{dep}
}\right) \bigg|_0^t +
\\
&\qquad \frac{
	\timescale{sfh}\timescale{dep}\timescale{rise}
}{
	\timescale{sfh}\timescale{rise} - \timescale{dep}\timescale{rise} -
	\timescale{sfh}\timescale{dep}
} \exp\left(
t'\frac{
	\timescale{sfh}\timescale{rise} - \timescale{dep}\timescale{rise} -
	\timescale{sfh}\timescale{dep}
}{
	\timescale{sfh}\timescale{dep}\timescale{rise}
}
\right)\bigg|_0^t + C \bigg)
\end{split}
\\
\begin{split}
&= \frac{1}{1 - e^{-t / \timescale{rise}}}
\exp \left(-t\frac{
	\timescale{sfh} - \timescale{dep}
}{
	\timescale{sfh}\timescale{dep}
}\right) \frac{y_\alpha}{\tau_\star} \bigg[\frac{
	\timescale{sfh}\timescale{dep}
}{
	\timescale{sfh} - \timescale{dep}
} \left(
\exp\left(
t\frac{
	\timescale{sfh} - \timescale{dep}
}{
	\timescale{sfh}\timescale{dep}
}\right) - 1\right) +
\\
&\qquad \frac{
	\timescale{sfh}\timescale{dep}\timescale{rise}
}{
	\timescale{sfh}\timescale{rise} - \timescale{dep}\timescale{rise} -
	\timescale{sfh}\timescale{dep}
} \left(
\exp\left(t\frac{
	\timescale{sfh}\timescale{rise} - \timescale{dep}\timescale{rise} -
	\timescale{sfh}\timescale{dep}
}{
	\timescale{sfh}\timescale{dep}\timescale{rise}
}\right) - 1\right) + 
\\
&\qquad C\bigg]
\end{split}
\\
\begin{split}
&= \frac{1}{1 - e^{-t / \timescale{rise}}}
\left(\frac{y_\alpha}{\tau_\star}\right)
\bigg[\frac{
	\timescale{sfh}\timescale{dep}
}{
	\timescale{sfh} - \timescale{dep}
} \left(
1 - \exp\left(-t\frac{
	\timescale{sfh} - \timescale{dep}
}{
	\timescale{sfh}\timescale{dep}
}\right)
\right) +
\\
&\qquad \frac{
	\timescale{sfh}\timescale{dep}\timescale{rise}
}{
	\timescale{sfh}\timescale{rise} - \timescale{dep}\timescale{rise} -
	\timescale{sfh}\timescale{dep}
} \left(e^{-t / \timescale{rise}} -
\exp\left(-t
\frac{
	\timescale{sfh} - \timescale{dep}
}{
	\timescale{sfh}\timescale{dep}
}
\right)
\right) + C\bigg]
\end{split}
\\
\begin{split}
&= \frac{1}{1 - e^{-t / \timescale{rise}}}
\left(\frac{y_\alpha}{1 + \eta - r}\right)
\bigg[\frac{
	\timescale{sfh}
}{
	\timescale{sfh} - \timescale{dep}
} \left(
1 - \exp\left(-t\frac{
	\timescale{sfh} - \timescale{dep}
}{
	\timescale{sfh}\timescale{dep}
}\right)
\right) +
\\
&\qquad \frac{
	\timescale{sfh}\timescale{rise}
}{
	\timescale{sfh}\timescale{rise} - \timescale{dep}\timescale{rise} -
	\timescale{sfh}\timescale{dep}
} \left(e^{-t / \timescale{rise}} -
\exp\left(-t
\frac{
	\timescale{sfh} - \timescale{dep}
}{
	\timescale{sfh}\timescale{dep}
}
\right)
\right)\bigg].
\end{split}
\end{align}\end{subequations}
At this point, it is now apparent that at~$t = 0$, the exponential factors are
each equal to 1, and the term in brackets is equal to zero if and only if
$C = 0$, which then satisfies the boundary condition of~$Z_\alpha(t = 0) = 0$.
I have therefore omitted the term C in the final equality to arrive at the
final solution for~$Z_\alpha(t)$.
I note that the term~$1 / (1 - e^{-t / \timescale{rise}})$ is infinite at
$t = 0$, and thus any choice of integration constant~$C$ would technically
satisfy the boundary condition that~$Z_\alpha(t = 0) = 0$, though~$C \neq 0$
solutions are purely mathematical and the~$C = 0$ solution is physical.


\newpage
\bibliographystyle{mnras}
\bibliography{onezone-mdfs}

\end{document}
